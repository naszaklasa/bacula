%%
%%

\section*{Installing and Configuring SQLite}
\label{_ChapterStart33}
\index[general]{Installing and Configuring SQLite }
\index[general]{SQLite!Installing and Configuring }
\addcontentsline{toc}{section}{Installing and Configuring SQLite}

\subsection*{Installing and Configuring SQLite -- Phase I}
\index[general]{Phase I!Installing and Configuring SQLite -- }
\index[general]{Installing and Configuring SQLite -- Phase I }
\addcontentsline{toc}{subsection}{Installing and Configuring SQLite -- Phase
I}

If you use the {\bf ./configure \verb{--{with-sqlite} statement for configuring {\bf
Bacula}, you will need SQLite version 2.2.3 or later installed. Our standard
location (for the moment) for SQLite is in the dependency package {\bf
depkgs/sqlite-2.2.3}. Please note that the version will be updated as new
versions are available and tested. 

Installing and Configuring is quite easy. 

\begin{enumerate}
\item Download the Bacula dependency packages  
\item Detar it with something like:

   {\bf tar xvfz depkgs.tar.gz}  

Note, the above command requires GNU tar. If you do not  have GNU tar, a
command such as:

{\bf zcat depkgs.tar.gz | tar xvf -}

will probably accomplish the same thing. 

\item {\bf cd depkgs}

\item {\bf make sqlite}  

   \end{enumerate}

At this point, you should return to completing the installation of {\bf
Bacula}. 

Please note that the {\bf ./configure} used to build {\bf Bacula} will need to
include {\bf \verb{--{with-sqlite}. 

\subsection*{Installing and Configuring SQLite -- Phase II}
\label{phase2}
\index[general]{Phase II!Installing and Configuring SQLite -- }
\index[general]{Installing and Configuring SQLite -- Phase II }
\addcontentsline{toc}{subsection}{Installing and Configuring SQLite -- Phase
II}

This phase is done {\bf after} you have run the {\bf ./configure} command to
configure {\bf Bacula}. 

{\bf Bacula} will install scripts for manipulating the database (create,
delete, make tables etc) into the main installation directory. These files
will be of the form *\_bacula\_* (e.g. create\_bacula\_database). These files
are also available in the \lt{}bacula-src\gt{}/src/cats directory after
running ./configure. If you inspect create\_bacula\_database, you will see
that it calls create\_sqlite\_database. The *\_bacula\_* files are provided
for convenience. It doesn't matter what database you have chosen;
create\_bacula\_database will always create your database. 

At this point, you can create the SQLite database and tables: 

\begin{enumerate}
\item cd \lt{}install-directory\gt{}

   This directory contains the Bacula catalog  interface routines.  

\item ./make\_sqlite\_tables

   This script creates the SQLite database as well as the  tables used by {\bf
Bacula}. This script will be  automatically setup by the {\bf ./configure}
program  to create a database named {\bf bacula.db} in {\bf Bacula's}  working
directory. 
\end{enumerate}

\subsection*{Linking Bacula with SQLite}
\index[general]{SQLite!Linking Bacula with }
\index[general]{Linking Bacula with SQLite }
\addcontentsline{toc}{subsection}{Linking Bacula with SQLite}

If you have followed the above steps, this will all happen automatically and
the SQLite libraries will be linked into {\bf Bacula}. 

\subsection*{Testing SQLite}
\index[general]{SQLite!Testing }
\index[general]{Testing SQLite }
\addcontentsline{toc}{subsection}{Testing SQLite}

As of this date (20 March 2002), we have much less ``production'' experience
using SQLite than using MySQL. That said, we should note that SQLite has
performed flawlessly for us in all our testing. 

\subsection*{Re-initializing the Catalog Database}
\index[general]{Database!Re-initializing the Catalog }
\index[general]{Re-initializing the Catalog Database }
\addcontentsline{toc}{subsection}{Re-initializing the Catalog Database}

After you have done some initial testing with {\bf Bacula}, you will probably
want to re-initialize the catalog database and throw away all the test Jobs
that you ran. To do so, you can do the following: 

\footnotesize
\begin{verbatim}
  cd <install-directory>
  ./drop_sqlite_tables
  ./make_sqlite_tables
\end{verbatim}
\normalsize

Please note that all information in the database will be lost and you will be
starting from scratch. If you have written on any Volumes, you must write and
end of file mark on the volume so that Bacula can reuse it. Do so with: 

\footnotesize
\begin{verbatim}
   (stop Bacula or unmount the drive)
   mt -f /dev/nst0 rewind
   mt -f /dev/nst0 weof
\end{verbatim}
\normalsize

Where you should replace {\bf /dev/nst0} with the appropriate tape drive
device name for your machine. 
