%%
%%

\section*{Bacula Console}
\label{_ChapterStart23}
\index[general]{Console!Bacula }
\index[general]{Bacula Console }
\addcontentsline{toc}{section}{Bacula Console}

\subsection*{General}
\index[general]{General }
\addcontentsline{toc}{subsection}{General}

The {\bf Bacula Console} (sometimes called the User Agent) is a program that
allows the user or the System Administrator, to interact with the Bacula
Director daemon while the daemon is running. 

The current Bacula Console comes in two versions: a shell interface (TTY
style), and a GNOME GUI interface. Both permit the administrator or authorized
users to interact with Bacula. You can determine the status of a particular
job, examine the contents of the Catalog as well as perform certain tape
manipulations with the Console program. 

In addition, there is a wx-console built with wxWidgets that allows a graphic
restore of files. As of version 1.34.1 it is in an early stage of development,
but it quite useful. 

Since the Console program interacts with the Director by the network, your
Console and Director programs do not necessarily need to run on the same
machine. 

In fact, a certain minimal knowledge of the Console program is needed in order
for Bacula to be able to write on more than one tape, because when Bacula
requests a new tape, it waits until the user, via the Console program,
indicates that the new tape is mounted. 

\subsection*{Configuration}
\index[general]{Configuration }
\addcontentsline{toc}{subsection}{Configuration}

When the Console starts, it reads a standard Bacula configuration file named
{\bf bconsole.conf} or {\bf gnome-console.conf} in the case of the GNOME
Console version. This file allows default configuration of the Console, and at
the current time, the only Resource Record defined is the Director resource,
which gives the Console the name and address of the Director. For more
information on configuration of the Console program, please see the 
\ilink{Console Configuration File}{_ChapterStart36} Chapter of
this document. 

\subsection*{Running the Console Program}
\index[general]{Running the Console Program }
\index[general]{Program!Running the Console }
\addcontentsline{toc}{subsection}{Running the Console Program}

After launching the Console program (bconsole), it will prompt you for the
next command with an asterisk (*). (Note, in the GNOME version, the prompt is
not present; you simply enter the commands you want in the command text box at
the bottom of the screen.) Generally, for all commands, you can simply enter
the command name and the Console program will prompt you for the necessary
arguments. Alternatively, in most cases, you may enter the command followed by
arguments. The general format is: 

\footnotesize
\begin{verbatim}
 <command> <keyword1>[=<argument1>] <keyword2>[=<argument2>] ...
\end{verbatim}
\normalsize

where {\bf command} is one of the commands listed below; {\bf keyword} is one
of the keywords listed below (usually followed by an argument); and {\bf
argument} is the value. The command may be abbreviated to the shortest unique
form. If two commands have the same starting letters, the one that will be
selected is the one that appears first in the {\bf help} listing. If you want
the second command, simply spell out the full command. None of the keywords
following the command may be abbreviated. 

For example: 

\footnotesize
\begin{verbatim}
list files jobid=23
\end{verbatim}
\normalsize

will list all files saved for JobId 23. Or: 

\footnotesize
\begin{verbatim}
show pools
\end{verbatim}
\normalsize

will display all the Pool resource records. 

\subsection*{Stopping the Console Program}
\index[general]{Program!Stopping the Console }
\index[general]{Stopping the Console Program }
\addcontentsline{toc}{subsection}{Stopping the Console Program}

Normally, you simply enter {\bf quit} or {\bf exit} and the Console program
will terminate. However, it waits until the Director acknowledges the command.
If the Director is already doing a lengthy command (e.g. prune), it may take
some time. If you want to immediately terminate the Console program, enter the
{\bf .quit} command. 

There is currently no way to interrupt a Console command once issued (i.e.
ctl-C does not work). However, if you are at a prompt that is asking you to
select one of several possibilities and you would like to abort the command,
you can enter a period ({\bf .}), and in most cases, you will either be
returned to the main command prompt or if appropriate the previous prompt (in
the case of nested prompts). In a few places such as where it is asking for a
Volume name, the period will be taken to be the Volume name. In that case, you
will most likely be able to cancel at the next prompt. 
\label{list}

\subsection*{Alphabetic List of Console Commands}
\index[general]{Commands!Alphabetic List of Console }
\index[general]{Alphabetic List of Console Commands }
\addcontentsline{toc}{subsection}{Alphabetic List of Console Commands}

The following commands are currently implemented: 

\begin{description}
\item [{add [pool=\lt{}pool-name\gt{} storage=\lt{}storage\gt{}
   jobid=\lt{}JobId\gt{}]} ]
   \index[console]{add [pool }
This command is used to add Volumes to an existing Pool. The  Volume names
entered are placed in the Catalog and thus become  available for backup
operations. Normally, the {\bf label}  command is used rather than this
command because the {\bf label}  command labels the physical media (tape) and
does the equivalent of  the {\bf add} command. This command affects only the
Catalog and  not the physical media (data on Volumes). The physical media must
exist and be labeled before use (usually with the {\bf label}  command). This
command can, however, be useful if you wish to add  a number of Volumes to the
Pool that will be physically labeled at  a later time. It can also be useful
if you are importing a tape  from another site. Please see the {\bf label}
command below for  the list of legal characters in a Volume name.  

\item [autodisplay on/off]
   \index[console]{autodisplay on/off }
   This command accepts {\bf on} or  {\bf off} as an argument, and turns
auto-display of messages on or  off respectively. The default for the console
program is  {\bf off}, which means that you will be notified when there are 
console messages pending, but they will not automatically be  displayed. The
default for the gnome-console program is  {\bf on}, which means that messages
will be displayed when  they are received (usually within 5 seconds of them
being  generated).  

When autodisplay is turned off, you must explicitly  retrieve the messages
with the {\bf messages} command. When  autodisplay is turned on, the messages
will be displayed on the  console as they are received.  

\item [automount on/off]
   \index[console]{automount on/off }
   This command accepts {\bf on} or {\bf off} as  the argument, and turns
auto-mounting of the tape after a  {\bf label} command on or off respectively.
The default is  {\bf on}. If {\bf automount} is turned off, you must
explicitly  {\bf mount} the tape after a label command to use it.  

\item [{cancel [jobid=\lt{}number\gt{} job=\lt{}job-name\gt{}]}]
   \index[console]{cancel [jobid }
   This  command is used to cancel a job and accepts {\bf jobid=nnn} or  {\bf
job=xxx} as an argument where nnn is replaced by the JobId  and xxx is
replaced by the job name. If you do not specify a  keyword, the Console
program will prompt you with the names of all  the active jobs allowing you to
choose one.  

Once a Job is marked to be canceled, it may take a bit of time  (generally
within a minute) before it actually terminates,  depending on what operations
it is doing. 

\item [{ create [pool=\lt{}pool-name\gt{}]}]
   \index[console]{create [pool }
   This command is used to  create a Pool record in the database using the Pool
resource record  defined in the Director's configuration file. So in a sense,
this  command simply transfers the information from the Pool resource in  the
configuration file into the Catalog. Normally this command is  done
automatically for you when the Director starts providing the  Pool is
referenced within a Job resource. If you use this command  on an existing
Pool, it will automatically update the Catalog to  have the same information
as the Pool resource. After creating a  Pool, you will most likely use the
{\bf label} command to label  one or more volumes and add their names to the
Media database.  

When starting a Job, if Bacula determines that there is  no Pool record in the
database, but there is a Pool resource of the  appropriate name, it will
create it for you. If you want the Pool  record to appear in the database
immediately, simply use this  command to force it to be created. 

\item [{ delete [volume=\lt{}vol-name\gt{} pool=\lt{}pool-name\gt{}  job
   jobid=\lt{}id\gt{}] }]
   \index[console]{delete }
The delete command is used to delete  a Volume, Pool or Job record from the
Catalog as well as all  associated Volume records that were created. This
command operates  only on the Catalog database and has no effect on the actual
data  written to a Volume. This command can be dangerous and we strongly 
recommend that you do not use it unless you know what you are  doing. 

If the keyword {\bf Volume} appears on the command  line, the named Volume
will be deleted from the catalog, if the  keyword {\bf Pool} appears on the
command line, a Pool will be  deleted, and if the keyword {\bf Job} appears on
the command line,  a Job and all its associated records (File and JobMedia)
will be  deleted from the catalog. The full form of this command is:  

delete pool=\lt{}pool-name\gt{}

or  

delete volume=\lt{}volume-name\gt{} pool=\lt{}pool-name\gt{}  or  

delete JobId=\lt{}job-id\gt{} JobId=\lt{}job-id2\gt{} ...  or  

delete Job JobId=n,m,o-r,t ...  

The first form deletes a Pool record  from the catalog database. The second
form deletes a Volume record  from the specified pool in the catalog database.
The  third form delete the specified Job record from the catalog  database.
The last form deletes JobId records for JobIds n,m,o,p,  q,r, and t. When each
one of the n,m,... is, of course, a number.  
\label{estimate}

\item [estimate]
   \index[console]{estimate }
   Using this command, you can get an idea how  many files will be backed up, or
if you are unsure about  your Include statements in your FileSet, you can test
them  without doing an actual backup. The default is to assume  a Full backup.
However, you can override this by specifying  a {\bf level=Incremental} or
{\bf level=Differential}  on the command line. A Job name must  be specified
or you will be prompted for one,  and optionally a Client and FileSet may be
specified  on the command line. It then  contacts the client which computes
the number of files and  bytes that would be backed up. Please note that this
is  an estimated calculated from the number of blocks in the  file rather than
by reading the actual bytes. As such, the  estimated backup size will
generally be larger than  an actual backup.  

Optionally you may specify the keyword {\bf listing} in  which case, all the
files to be backed up will be listed.  Note, it could take quite some time to
display them if the  backup is large. The full form is:  

estimate job=\lt{}job-name\gt{} listing client=\lt{}client-name\gt{} 
fileset=\lt{}fileset-name\gt{} level=\lt{}level-name\gt{}  

Specification of the {\bf job} is sufficient, but you can also  override the
client, fileset and/or level  by specifying them on the estimate command line.
 

As an example, you might do:  

\footnotesize
\begin{verbatim}
     @output /tmp/listing
     estimate job=NightlySave listing level=Incremental
     @output
     
\end{verbatim}
\normalsize

which will do a full listing of all files to be backed up for the  Job {\bf
NightlySave} during an Incremental save and put it in the  file {\bf
/tmp/listing}. 

\item [help]
   \index[console]{help }
   This command displays the list of commands available.  

\item [label]
   \index[console]{label }
   This command is used to label physical volumes.  The full form of this command
is:

label storage=\lt{}storage-name\gt{} volume=\lt{}volume-name\gt{}
slot=\lt{}slot\gt{}  

If you leave out any part, you will be prompted for it.  The media type is
automatically taken from the Storage resource  definition that you supply.
Once the necessary information is  obtained, the Console program contacts the
specified Storage  daemon and requests that the tape be labeled. If the tape
labeling  is successful, the Console program will create a Volume record  in
the appropriate Pool.  

The Volume name is restricted to letters, numbers, and  the special characters
hyphen ({\bf -}), underscore ({\bf \_}),  colon ({\bf :}), and period ({\bf
.}).  All other characters including a space are illegal.  This restriction is
to ensure good readability of Volume names  to reduce operator errors.  

Please note, when labeling a blank tape, Bacula will get  read I/O error when
it attempts to ensure that the tape is already  labeled. If you wish to avoid
getting these messages, please write  and EOF mark on your tape before
attempting to label it:  

\footnotesize
\begin{verbatim}
       mt rewind
       mt weof
       
\end{verbatim}
\normalsize

The label command can fail for a number of reasons:  

   \begin{enumerate}
   \item The Volume name you specify is already in the  Volume database.  
   \item The Storage daemon has a tape already mounted on the  device, in which
   case you must {\bf unmount}  the device, insert a blank tape, then do the 
   {\bf label} command.  
   \item The tape in the device is already a Bacula labeled tape.  (Bacula will
   never relabel a Bacula labeled tape unless  it is recycled and you use the
   {\bf relabel} command).  
   \item There is no tape in the drive.  
   \end{enumerate}

There are two ways to relabel a volume that already has  a Bacula label. The
brute  force method is to write an end of file mark on the tape  using the
system {\bf mt} program, something like the  following:  

\footnotesize
\begin{verbatim}
       mt -f /dev/st0 rewind
       mt -f /dev/st0 weof
       
\end{verbatim}
\normalsize

Then you use the {\bf label} command to add a new label.  However, this could
leave traces of the old volume in the  catalog.  

The preferable method to relabel a tape is to first {\bf purge}  the volume,
either automatically, or explicitly with the  {\bf purge} command, then use
the {\bf relabel} command described  below.  

If your autochanger has barcode labels, you can label all the  Volumes in your
autochanger one after another by using the  {\bf label barcodes} command. For
each tape in the changer containing  a barcode, Bacula will mount the tape and
then label it with the  same name as the barcode. An appropriate Media record
will also be  created in the catalog. Any barcode that begins with the same
characters  as specified on the ``CleaningPrefix=xxx'' command, will be
treated as a  cleaning tape, and will not be labeled. For example with:  

\footnotesize
\begin{verbatim}
        Pool {
          Name ...
          Cleaning Prefix = "CLN"
        }
        
\end{verbatim}
\normalsize

Any slot containing a barcode of CLNxxxx will be treated as a cleaning  tape
and will not be mounted. Note, the full form of the command is: 

\footnotesize
\begin{verbatim}
     
update storage=xxx pool=yyy slots=1-5,10 barcodes
\end{verbatim}
\normalsize

\item [list]
   \index[console]{list }
   The list command lists the requested contents of the  Catalog. The various
fields of each record are listed on a single  line. If there The various forms
of the list command are:

list jobs

list jobid=\lt{}id\gt{}

list job=\lt{}job-name\gt{}

list jobmedia

list jobmedia jobid=\lt{}id\gt{}

list jobmedia job=\lt{}job-name\gt{}

list files jobid=\lt{}id\gt{}

list files job=\lt{}job-name\gt{}

list pools

list clients

list jobtotals

list volumes

list volumes jobid=\lt{}id\gt{}

list volumes pool=\lt{}pool-name\gt{}

list volumes job=\lt{}job-name\gt{}

list volume=\lt{}volume-name\gt{}  list nextvolume job=\lt{}job-name\gt{}

list nextvol job=\lt{}job-name\gt{}

What most of the above commands do should be more or  less obvious. In general
if you do not specify all  the command line arguments, the command will prompt
you  for what is needed.  

The {\bf list nextvol} command will print the Volume  name to be used by the
specified job. You should be aware  that exactly what Volume will be used
depends on a lot  of factors including the time and what a prior job  will do.
It may fill a tape that is not full when  you issue this command. As a
consequence, this command  will give you a good estimate of what Volume will
be  used but not a definitive answer. In addition, this  command may have
certain side effect because it  runs through the same algorithm as a job, 
which means it may automatically purge or recycle a  Volume.  

If you wish to add specialized commands that list the contents  of the
catalog, you can do so by adding them to the  {\bf query.sql} file. However,
this takes some knowledge  of programming SQL. Please see the {\bf query}
command below for  additional information. See below for listing the full
contents  of a catalog record with the {\bf llist} command.  

As an example, the command {\bf list pools} might produce  the following
output: 

\footnotesize
\begin{verbatim}
+------+---------+---------+---------+----------+-------------+
| PoId | Name    | NumVols | MaxVols | PoolType | LabelFormat |
+------+---------+---------+---------+----------+-------------+
|    1 | Default |       0 |       0 | Backup   | *           |
|    2 | Recycle |       0 |       8 | Backup   | File        |
+------+---------+---------+---------+----------+-------------+
\end{verbatim}
\normalsize

As mentioned above, the {\bf list} command lists what is in the  database.
Some things are put into the database immediately when  Bacula starts up, but
in general, most things are put in only when  they are first used, which is
the case for a Client as with Job  records, etc.  

Bacula should create a client record in the database the first  time you run a
job for that client. Doing a {\bf status} will not  cause a database record to
be created. The client database record  will be created whether or not job
fails, but it must at least  start. When the Client is actually contacted,
additional info  from the client will be added to the client record (a ``uname
-a''  output).  

If you want to see what Client resources you have available in  your conf
file, you use the Console command {\bf show clients}.  

\item [llist]
   \index[console]{llist }
   The llist or ``long list'' command takes  all the same arguments that the list
command described above does.  The difference is that the llist command list
the full contents  of each database record selected. It does so by listing the
various fields of the record vertically, with one field per  line. It is
possible to produce a very large number of output  lines with this command.  

If instead of the {\bf list pools} as in the example above,  you enter {\bf
llist pools} you might get the following output:  

\footnotesize
\begin{verbatim}
          PoolId: 1
            Name: Default
         NumVols: 0
         MaxVols: 0
         UseOnce: 0
      UseCatalog: 1
 AcceptAnyVolume: 1
    VolRetention: 1,296,000
  VolUseDuration: 86,400
      MaxVolJobs: 0
     MaxVolBytes: 0
       AutoPrune: 0
         Recycle: 1
        PoolType: Backup
     LabelFormat: *
          PoolId: 2
            Name: Recycle
         NumVols: 0
         MaxVols: 8
         UseOnce: 0
      UseCatalog: 1
 AcceptAnyVolume: 1
    VolRetention: 3,600
  VolUseDuration: 3,600
      MaxVolJobs: 1
     MaxVolBytes: 0
       AutoPrune: 0
         Recycle: 1
        PoolType: Backup
     LabelFormat: File
      
\end{verbatim}
\normalsize

\item [messages]
   \index[console]{messages }
   This command causes any pending  console messages to be immediately displayed.
 

\item [mount]
   \index[console]{mount }
   The mount command is used to get Bacula to  read a volume on a physical
device. It is a way to tell  Bacula that you have mounted a tape and that
Bacula should  examine the tape. This command is used only after there was  no
Volume in a drive and Bacula requests you to mount a new  Volume or when you
have specifically unmounted a Volume with  the {\bf unmount} console command,
which causes Bacula to  close the drive. If you have an autoloader, the mount 
command will not cause Bacula to operate the autoloader. The  various forms of
the mount command are:

mount  storage=\lt{}storage-name\gt{}

mount [ jobid=\lt{}id\gt{} |  job=\lt{}job-name\gt{} ]

If you have specified {\bf Automatic  Mount = yes} in the Storage daemon's
Device resource,  under most circumstances, Bacula will automatically access 
the Volume unless you have explicitly {\bf unmount}ed it in  the Console
program. 
\label{ManualPruning}

\item [prune]
   \index[console]{prune }
   The Prune command allows you to safely  remove expired database records from
Jobs and Volumes.  This command works only on the Catalog database and does
not  affect data written to Volumes. In all  cases, the Prune command applies
a retention period to the  specified records. You can Prune expired File
entries from  Job records; you can Prune expired Job records from the 
database, and you can Prune both expired Job and File records  from specified
Volumes.  

prune files|jobs|volume client=\lt{}client-name\gt{} 
volume=\lt{}volume-name\gt{}  

For a Volume to be pruned, the {\bf VolStatus}  must be Full, Used, or Append,
otherwise the pruning will not  take place.  

\item [purge]
   \index[console]{purge }
   The Purge command will delete associated  Catalog database  records from Jobs
and Volumes without considering the  retention period. {\bf Purge} works only
on the Catalog database  and does not affect data written to Volumes.  This
command can be dangerous because you  can delete catalog records associated
with current backups of  files, and we recommend  that you do not use it
unless you know what you are doing.  The permitted forms of {\bf purge} are: 
purge files
jobid=\lt{}jobid\gt{}|job=\lt{}job-name\gt{}|client=\lt{}client-name\gt{} 

purge jobs client=\lt{}client-name\gt{} (of all jobs)

purge volume|volume=\lt{}vol-name\gt{} (of all jobs)

For the {\bf purge} command to work on Volume Catalog database  records the
{\bf VolStatus}  must be Append, Full, Used, or Error.  

The actual data written to the Volume will be unaffected by  this command.  

\item [relabel]
   \index[console]{relabel }
   This command is used to label physical volumes.  The full form of this command
is:

relabel storage=\lt{}storage-name\gt{} volume=\lt{}newvolume-name\gt{} 
name=\lt{}old-volume-name\gt{}  

If you leave out any part, you will be prompted for it.  In order for the
Volume (old-volume-name) to be relabeled,  it must be in the catalog, and the
volume status must  be marked {\bf Purged} or {\bf Recycle}. This happens
automatically as a  result of applying retention periods, or you may
explicitly  purge the volume using the {\bf purge} command.  

Once the volume is physically relabeled, the old data written on  the Volume
is lost and cannot be recovered.  

\item [release]
   \index[console]{release }
   This command is used to cause the Storage  daemon to rewind (release) the
current tape in the drive, and  to re-read the Volume label the next time the
tape is used.  

release storage=\lt{}storage-name\gt{}  

After a release command, the device is still kept open  by Bacula (unless
Always Open is set to No in the Storage  Daemon's configuration) so it cannot
be used by another program.  However, with some tape drives,  the operator can
remove the current tape and to insert a  different one, and when the next Job
starts, Bacula will  know to re-read the tape label to find out what tape is 
mounted. If you want to be able to use the drive with  another program (e.g.
{\bf mt}), you must use the {\bf unmount}  command to cause Bacula to
completely release (close) the device.  

\item [restore]
   \index[console]{restore }
   The restore command allows you to select one  or more Jobs (JobIds) to be
restored using various methods.  Once the JobIds are selected, the File
records  for those Jobs are placed in an internal Bacula directory  tree, and
the restore enters a file selection mode that allows  you to interactively
walk up and down the file tree selecting  individual files to be restored.
This mode is somewhat similar to  the standard Unix {\bf restore} program's
interactive file  selection mode.  

restore storage=\lt{}storage-name\gt{} client=\lt{}client-name\gt{} 
where=\lt{}path\gt{} pool=\lt{}pool-name\gt{} fileset=\lt{}fileset-name\gt{} 
select current all done  

Where {\bf current}, if specified, tells the restore command  to automatically
select a restore to the most current backup.  If not specified, you will be
prompted. The {\bf all} specification  tells the restore command to restore
all files. If it is not  specified, you will be prompted for the files to
restore.  For details of the {\bf restore} command, please see the  
\ilink{Restore Chapter}{_ChapterStart13} of this manual.  

\item [run]
   \index[console]{run }
   This command allows you to schedule jobs  to be run immediately. The full form
of the command is:

run job=\lt{}job-name\gt{} client=\lt{}client-name\gt{}
fileset=\lt{}FileSet-name\gt{}  level=\lt{}level-keyword\gt{}
storage=\lt{}storage-name\gt{}  where=\lt{}directory-prefix\gt{}
when=\lt{}universal-time-specification\gt{}  yes  

Any information that is needed but not specified will be  listed for
selection, and before starting the job, you will  be prompted to accept,
reject, or modify the parameters of  the job to be run, unless you have
specified {\bf yes}, in  which case the job will be immediately sent to the
scheduler.  

On my system, when I enter a run command, I get the following  prompt:  

\footnotesize
\begin{verbatim}
A job name must be specified.
The defined Job resources are:
     1: Matou
     2: Polymatou
     3: Rufus
     4: Minimatou
     5: Minou
     6: PmatouVerify
     7: MatouVerify
     8: RufusVerify
     9: Watchdog
Select Job resource (1-9):
     
\end{verbatim}
\normalsize

If I then select number 5, I am prompted with:  

\footnotesize
\begin{verbatim}
Run Backup job
JobName:  Minou
FileSet:  Minou Full Set
Level:    Incremental
Client:   Minou
Storage:  DLTDrive
Pool:     Default
When:     2003-04-23 17:08:18
OK to run? (yes/mod/no):
     
\end{verbatim}
\normalsize

If I now enter {\bf yes}, the Job will be run. If I enter {\bf mod},  I will
be presented with the following prompt.  

\footnotesize
\begin{verbatim}
Parameters to modify:
     1: Level
     2: Storage
     3: Job
     4: FileSet
     5: Client
     6: When
     7: Pool
Select parameter to modify (1-7):
     
\end{verbatim}
\normalsize

If you wish to start a job at a later time, you can do so by setting  the When
time. Use the {\bf mod} option and select {\bf When} (no. 6).  Then enter the
desired start time in YYYY-MM-DD HH:MM:SS format.  

\item [setdebug]
   \index[dir]{setdebug }
   This command is used to set the debug level in each  daemon. The form of this
command is:

setdebug level=nn [trace=0/1 client=\lt{}client-name\gt{} | dir | director |
storage=\lt{}storage-name\gt{} | all]  

If trace=1 is set, then the tracing will be enabled, and the  daemon where the
setdebug applies will be placed in trace mode, and  all debug output will go
to the file {\bf bacula.trace} in the  current directory of the daemon.
Normally, tracing is used only for  Win32 clients where the debug output
cannot be written to a terminal  or redirected to a file. When tracing, each
debug output message is  appended to the trace file. You must explicitly
delete the file when  you are done.  

\item [show]
   \index[console]{show }
   The show command will  list the Director's resource records as defined in  the
Director's configuration file (normally {\bf bacula-dir.conf}).  This command
is used mainly for debugging purposes by developers.  The following keywords
are accepted on the show command line:  directors, clients, counters, jobs,
storages, catalogs,  schedules, filesets, groups, pools, messages, all, help. 
Please don't confuse this command with the {\bf list}, which displays  the
contents of the catalog.  

\item [sqlquery]
   \index[dir]{sqlquery }
   The sqlquery command puts the Console program  into SQL query mode where each
line you enter is concatenated  to the previous line until a semicolon (;) is
seen. The semicolon  terminates the command, which is then passed directly  to
the SQL database engine. When the output from the SQL engine  is displayed,
the formation of a new SQL command begins. To  terminate SQL query mode and
return to the Console command  prompt, you enter a period (.) in column 1.

Using this command, you can query the SQL catalog database  directly. Note you
should really know what you are doing otherwise  you could damage the catalog
database. See the {\bf query} command below  for simpler and safer way of
entering SQL queries.  

Depending on what database engine you are using (MySQL or SQLite),  you will
have somewhat different SQL commands available. For more  detailed
information, please refer to the MySQL or SQLite documentation.  

\item [status]
   \index[dir]{status }
   This command will display the status of the next  jobs that are scheduled
during the next twenty-four hours as  well as the status of currently running
jobs. The full form  of this command is:  

status [all | dir=\lt{}dir-name\gt{} | director | 
client=\lt{}client-name\gt{} | storage=\lt{}storage-name\gt{}]  

If you do a {\bf status dir}, the console will list  any currently running
jobs, a summary of all  jobs scheduled to be run in the next 24 hours, and  a
listing of the last 10 terminated jobs with their  statuses. The scheduled
jobs summary  will include the Volume name to be used. You should be aware  of
two things: 1. to obtain the volume name, the code  goes through the same code
that will be used when the  job runs, which means that it may prune or recycle
a Volume;  2. The Volume listed is only a best guess. The Volume  actually
used may be different because of the time  difference (more durations may
expire when the job  runs) and another job could completely fill the Volume 
requiring a new one.  

In the Running Jobs listing, you may find the following  types of information:


\footnotesize
\begin{verbatim}
2507 Catalog MatouVerify.2004-03-13_05.05.02 is waiting execution
5349 Full    CatalogBackup.2004-03-13_01.10.00 is waiting for higher
             priority jobs to finish
5348 Differe Minou.2004-03-13_01.05.09 is waiting on max Storage jobs
5343 Full    Rufus.2004-03-13_01.05.04 is running
\end{verbatim}
\normalsize

Looking at the above listing from bottom to top,  obviously JobId 5343 (Rufus)
is running. JobId 5348  (Minou) is waiting for JobId 5343 to finish because it
is using the  Storage resource, hence the ``waiting on max Storage jobs''.
JobId  5349 has a lower priority than all the other jobs so it is waiting for 
higher priority jobs to finish, and finally, JobId 2508 (MatouVerify)  is
waiting because only one job can run at a time, hence it is simply  ``waiting
execution\&quot.</dd>  

\item [unmount]
   \index[console]{unmount }
   This command causes the indicated Bacula Storage  daemon to unmount the
specified device. The forms of the command  are the same as the mount command:

unmount storage=\lt{}storage-name\gt{}

unmount [ jobid=\lt{}id\gt{} | job=\lt{}job-name\gt{} ]

\item [update]
   \index[console]{update }
   This command will update catalog for  either a specific Pool record, a Volume
record, or the Slots in an  autochanger with barcode capability. In the case
of updating a  Pool record, the new information will be automatically taken
from  the corresponding Director's configuration resource record. It  can be
used to increase the maximum number of volumes permitted or  to set a maximum
number of volumes. The following main  keywords may be specified:  

media, volume, pool, slots  

In the case of updating a  Volume, you will be prompted for which value you
wish to change.  The following Volume parameters may be changed:  

\footnotesize
\begin{verbatim}
 
         Volume Status
         Volume Retention Period
         Volume Use Duration
         Maximum Volume Jobs
         Maximum Volume Files
         Maximum Volume Bytes
         Recycle Flag
         Slot
         InChanger Flag
         Pool
         Volume Files
         Volume from Pool
         All Volumes from Pool
      
\end{verbatim}
\normalsize

For slots {\bf update slots}, Bacula will obtain a  list of slots and their
barcodes from the Storage daemon,  and for each barcode found, it will
automatically update the  slot in the catalog Media record to correspond to
the new value.  This is very useful if you have moved cassettes in the
magazine,  or if you have removed the magazine and inserted a different  one.
As the slot of each Volume is updated, the InChanger flag for  that Volume
will also be set, and any other Volumes in the Pool  will have their InChanger
flag turned off. This permits Bacula to  know what magazine (tape holder) is
currently in the autochanger.  

If you do not have barcodes, you can accomplish the same thing  in version
1.33 and later by using the {\bf update slots scan}  command. The {\bf scan}
keyword tells Bacula to physically mount  each tape and to read its
VolumeName.  

For Pool {\bf update pool}, Bacula will move  the Volume record from its
existing poole to the pool specified.  

For {\bf Volume from Pool} and {\bf All Volumes from Pool},  the following
values are updated from the Pool record:  Recycle, VolRetention,
VolUseDuration, MaxVolJobs, MaxVolFiles,  and MaxVolBytes.  

The full form of the update command with all command line  arguments is:  

\footnotesize
\begin{verbatim}
       update volume=xxx pool=yyy slots volstatus=xxx VolRetention=ddd
         VolUse=ddd MaxVolJobs=nnn MaxVolBytes=nnn Recycle=yes|no
         slot=nnn
      
\end{verbatim}
\normalsize

\item [use]
   \index[console]{use }
   This command allows you to specify which Catalog  database to use. Normally,
you will be using only one database so  this will be done automatically. In
the case that you are using  more than one database, you can use this command
to switch from  one to another.  

use \lt{}database-name\gt{} 

\item [var]
   \label{var}
   \index[console]{var name }
This command takes a string or quoted string and  does variable expansion on
it the same way variable expansion  is done on the {\bf LabelFormat} string.
Thus, for the  most part, you can test your LabelFormat strings. The
difference  between the {\bf var} command and the actual LabelFormat process 
is that during the var command, no job is running so ''dummy``  values are
used in place of Job specific variables. Generally,  however, you will get a
good idea of what is going to happen  in the real case.  

\item [version]
   \index[console]{version }
   The command prints the Director's version.  

\item [quit]
   \index[console]{quit }
   This command terminates the console program. The  console program sends the
{\bf quit} request to the Director  and waits for acknowledgment. If the
Director is busy doing  a previous command for you that has not terminated, it
may  take some time. You may quit immediately by issuing the  {\bf .quit}
command (i.e. quit preceded by a period).  

\item [query]
   \index[console]{query }
   This command reads a predefined SQL query from  the query file (the name and
location of the  query file is defined with the QueryFile resource record in 
the Director's configuration file). You are prompted to select  a query from
the file, and possibly enter one or more parameters,  then the command is
submitted to the Catalog database SQL engine.  

The following queries are currently available (version 1.24):  

\footnotesize
\begin{verbatim}
Available queries:
  1: List Job totals:
  2: List where a file is saved:
  3: List where the most recent copies of a file are saved:
  4: List total files/bytes by Job:
  5: List total files/bytes by Volume:
  6: List last 20 Full Backups for a Client:
  7: List Volumes used by selected JobId:
  8: List Volumes to Restore All Files:
  9: List where a File is saved:
Choose a query (1-9):
      
\end{verbatim}
\normalsize

\item [exit]
   \index[console]{exit }
   This command terminates the console program.  

\item [wait]
   \index[console]{wait }
   The wait command causes the Director to pause  until there are no jobs
running. This command is useful in  a batch situation such as regression
testing where you  wish to start a job and wait until that job completes 
before continuing. 
\end{description}

\label{dotcommands}

\subsection*{Special dot Commands}
\index[general]{Commands!Special dot }
\index[general]{Special dot Commands }
\addcontentsline{toc}{subsection}{Special dot Commands}

There is a list of commands that are prefixed with a period (.). These
commands are intended to be used either by batch programs or graphical user
interface front-ends. They are not normally used by interactive users. Once
GUI development begins, this list will be considerably expanded. The following
is the list of dot commands: 

\footnotesize
\begin{verbatim}
.die         cause the Director to segment fault (for debugging)
.jobs        list all job names
.filesets    list all fileset names
.clients     list all client names
.msgs        return any queued messages
.quit        quit
.exit        quit
\end{verbatim}
\normalsize

\label{atcommands}

\subsection*{Special At (@) Commands}
\index[general]{Commands!Special At @ }
\index[general]{Special At (@) Commands }
\addcontentsline{toc}{subsection}{Special At (@) Commands}

Normally, all commands entered to the Console program are immediately
forwarded to the Director, which may be on another machine, to be executed.
However, there is a small list of {\bf at} commands, all beginning with a at
character (@), that will not be sent to the Director, but rather interpreted
by the Console program directly. Note, these commands are implemented only in
the tty console program and not in the GNOME Console. These commands are: 

\begin{description}

\item [@input \lt{}filename\gt{}]
   \index[console]{@input \lt{}filename\gt{} }
   Read and execute the commands  contained in the file specified.  

\item [@output \lt{}filename\gt{} w/a]
   \index[console]{@output \lt{}filename\gt{} w/a }
   Send all following output to the  filename specified either overwriting the
file (w) or appending to  the file (a). To redirect the output to the
terminal, simply enter  {\bf @output} without a filename specification.
WARNING: be careful  not to overwrite a valid file. A typical example during a
regression  test might be:  

\footnotesize
\begin{verbatim}
    @output /dev/null
    commands ...
    @output
    
\end{verbatim}
\normalsize

\item [@tee \lt{}filename\gt{} w/a]
   \index[console]{@tee \lt{}filename\gt{} w/a }
   Send all subsequent output to  both the specified file and the terminal. It is
turned off by  specifying {\bf @tee} or {\bf @output} without a filename.  

\item [@sleep \lt{}seconds\gt{}]
   \index[console]{@sleep \lt{}seconds\gt{} }
   Sleep the specified number of seconds.  

\item [@time]
   \index[console]{@time }
   Print the current time and date.  

\item [@version]
   \index[console]{@version }
   Print the console's version.  

\item [@quit]
   \index[console]{@quit }
   quit  

\item [@exit]
   \index[console]{@exit }
   quit  

\item [@\# anything]
   \index[console]{anything }
   Comment 
\end{description}

\label{scripting}

\subsection*{Running the Console Program from a Shell Script}
\index[general]{Script!Running the Console Program from a Shell }
\index[general]{Running the Console Program from a Shell Script }
\addcontentsline{toc}{subsection}{Running the Console Program from a Shell
Script}

You can automate many Console tasks by running the console program from a
shell script. For example, if you have created a file containing the following
commands: 

\footnotesize
\begin{verbatim}
 ./bconsole -c ./bconsole.conf <<END_OF_DATA
 unmount storage=DDS-4
 quit
 END_OF_DATA
\end{verbatim}
\normalsize

when that file is executed, it will unmount the current DDS-4 storage device.
You might want to run this command during a Job by using the {\bf
RunBeforeJob} or {\bf RunAfterJob} records. 

It is also possible to run the Console program from file input where the file
contains the commands as follows: 

\footnotesize
\begin{verbatim}
./bconsole -c ./bconsole.conf <filename
\end{verbatim}
\normalsize

where the file named {\bf filename} contains any set of console commands. 

As a real example, the following script is part of the Bacula regression
tests. It labels a volume (a disk volume), runs a backup, then does a restore
of the files saved. 

\footnotesize
\begin{verbatim}
bin/bconsole -c bin/bconsole.conf <<END_OF_DATA
@output /dev/null
messages
@output /tmp/log1.out
label volume=TestVolume001
run job=Client1 yes
wait
messages
@#
@# now do a restore
@#
@output /tmp/log2.out
restore current all
yes
wait
messages
@output
quit
END_OF_DATA
\end{verbatim}
\normalsize

The output from the backup is directed to /tmp/log1.out and the output from
the restore is directed to /tmp/log2.out. To ensure that the backup and
restore ran correctly, the output files are checked with: 

\footnotesize
\begin{verbatim}
grep "^Termination: *Backup OK" /tmp/log1.out
backupstat=$?
grep "^Termination: *Restore OK" /tmp/log2.out
restorestat=$?
\end{verbatim}
\normalsize

\subsection*{Adding Volumes to a Pool}
\index[general]{Adding Volumes to a Pool }
\index[general]{Pool!Adding Volumes to a }
\addcontentsline{toc}{subsection}{Adding Volumes to a Pool}

If you have used the {\bf label} command to label a Volume, it will be
automatically added to the Pool, and you will not need to add any media to the
pool. 

Alternatively, you may choose to add a number of Volumes to the pool without
labeling them. At a later time when the Volume is requested by {\bf Bacula}
you will need to label it. 

Before adding a volume, you must know the following information: 

\begin{enumerate}
\item The name of the Pool (normally ''Default``)  
\item The Media Type as specified in the Storage Resource  in the Director's
   configuration file (e.g. ''DLT8000``)  
\item The number and names of the Volumes you wish to create. 
   \end{enumerate}

For example, to add media to a Pool, you would issue the following commands to
the console program: 

\footnotesize
\begin{verbatim}
*add
Enter name of Pool to add Volumes to: Default
Enter the Media Type: DLT8000
Enter number of Media volumes to create. Max=1000: 10
Enter base volume name: Save
Enter the starting number: 1
10 Volumes created in pool Default
*
\end{verbatim}
\normalsize

To see what you have added, enter: 

\footnotesize
\begin{verbatim}
*list media pool=Default
+-------+----------+---------+---------+-------+------------------+
| MedId | VolumeNa | MediaTyp| VolStat | Bytes | LastWritten      |
+-------+----------+---------+---------+-------+------------------+
|    11 | Save0001 | DLT8000 | Append  |     0 | 0000-00-00 00:00 |
|    12 | Save0002 | DLT8000 | Append  |     0 | 0000-00-00 00:00 |
|    13 | Save0003 | DLT8000 | Append  |     0 | 0000-00-00 00:00 |
|    14 | Save0004 | DLT8000 | Append  |     0 | 0000-00-00 00:00 |
|    15 | Save0005 | DLT8000 | Append  |     0 | 0000-00-00 00:00 |
|    16 | Save0006 | DLT8000 | Append  |     0 | 0000-00-00 00:00 |
|    17 | Save0007 | DLT8000 | Append  |     0 | 0000-00-00 00:00 |
|    18 | Save0008 | DLT8000 | Append  |     0 | 0000-00-00 00:00 |
|    19 | Save0009 | DLT8000 | Append  |     0 | 0000-00-00 00:00 |
|    20 | Save0010 | DLT8000 | Append  |     0 | 0000-00-00 00:00 |
+-------+----------+---------+---------+-------+------------------+
*
\end{verbatim}
\normalsize

Notice that the console program automatically appended a number to the base
Volume name that you specify (Save in this case). If you don't want it to
append a number, you can simply answer 0 (zero) to the question ''Enter number
of Media volumes to create. Max=1000:``, and in this case, it will create a
single Volume with the exact name you specify. 
