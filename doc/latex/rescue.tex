%%
%%

\section*{Disaster Recovery Using Bacula}
\label{_ChapterStart38}
\index[general]{Disaster Recovery Using Bacula }
\index[general]{Bacula!Disaster Recovery Using }
\addcontentsline{toc}{section}{Disaster Recovery Using Bacula}

\subsection*{General}
\index[general]{General }
\addcontentsline{toc}{subsection}{General}

When disaster strikes, you must have a plan, and you must have prepared in
advance otherwise the work of recovering your system and your files will be
considerably greater. For example, if you have not previously saved the
partitioning information for your hard disk, how can you properly rebuild it
if the disk must be replaced? 

Unfortunately, many of the steps one must take before and immediately after a
disaster are very operating system dependent. As a consequence, this chapter
will discuss in detail disaster recovery (also called Bare Metal Recovery) for
{\bf Linux} and {\bf Solaris}. For Solaris, the procedures are still quite
manual. For FreeBSD the same procedures may be used but they are not yet
developed. For Win32, no luck. Apparently an ``emergency boot'' disk allowing
access to the full system API without interference does not exist. 
\label{considerations1}

\subsection*{Important Considerations}
\index[general]{Important Considerations }
\index[general]{Considerations!Important }
\addcontentsline{toc}{subsection}{Important Considerations}

Here are a few important considerations concerning disaster recovery that you
should take into account before a disaster strikes. 

\begin{itemize}
\item If the building which houses your computers burns down or is otherwise 
   destroyed, do you have off-site backup data? 
\item Disaster recovery is much easier if you have several machines. If  you
   have a single machine, how will you handle unforeseen events  if your only
   machine is down? 
\item Do you want to protect your whole system and use Bacula to  recover
   everything? or do you want to try to restore your system from  the original
   installation disks and apply any other updates and  only restore user files? 
\end{itemize}

\label{steps1}

\subsection*{Steps to Take Before Disaster Strikes}
\index[general]{Steps to Take Before Disaster Strikes }
\index[general]{Strikes!Steps to Take Before Disaster }
\addcontentsline{toc}{subsection}{Steps to Take Before Disaster Strikes}

\begin{itemize}
\item Create a Bacula Rescue CDROM for each of your Linux systems. Note,  it
   is possible to create one CDROM by copying the bacula-hostname  directory from
   each machine to the machine where you will be  burning the CDROM.  
\item Ensure that you always have a valid bootstrap file for your  backup that
   is saved to an alternate machine. This will permit  you to easily do a full
   restore of your system. 
\item If possible copy your catalog nightly to an alternate machine.  If you
   have a valid bootstrap file, this is not necessary, but  can be very useful if
   you do not want to reload everything. .  
\item Ensure that you always have a valid bootstrap file for your  catalog
   backup that is saved to an alternate machine. This will  permit you to restore
   your catalog more easily if needed.  
\item Test using the Bacula Rescue CDROM before you are forced to use  it in
   an emergency situation. 
   \end{itemize}

\label{rescueCDROM}

\subsection*{Bare Metal Recovery on Linux with a Bacula Rescue CDROM}
\index[general]{Bare Metal Recovery on Linux with a Bacula Rescue CDROM }
\index[general]{CDROM!Bare Metal Recovery on Linux with a Bacula Rescue }
\addcontentsline{toc}{subsection}{Bare Metal Recovery on Linux with a Bacula
Rescue CDROM}

The remainder of this section concerns recovering a {\bf Linux} computer, and
parts of it relate to the Red Hat version of Linux. The {\bf Solaris}
procedures can be found below under the 
\ilink{Solaris Bare Metal Recovery}{solaris} section of this
chapter. 

If you wish to use a floppy for restoration, please see the chapter 
\ilink{Bare Metal Floppy Recovery on Linux with a Bacula Floppy Rescue
Disk}{_ChapterStart24}, but be aware that the Bacula floppy disk is deprecated
and replaced by the CDROM rescue described in this chapter. 

A so called ``Bare Metal'' recovery is one where you start with an empty hard
disk and you restore your machine. There are also cases where you may lose a
file or a directory and want it restored. Please see the previous chapter for
more details for those cases. 

Bare Metal Recovery assumes that you have the following items for your system:

\begin{itemize}
\item A Bacula Rescue CDROM containing a copy of your OS and  a copy of your
   hard disk information, as well as a  statically linked version of the Bacula
   File daemon. This chapter describes how to build such a CDROM.
\item A full Bacula backup of your system possibly including  Incremental or
   Differential backups since the last Full  backup 
\item A second system running the Bacula Director, the Catalog,  and the
   Storage daemon. (this is not an absolute requirement,  but how to get around
   it is not yet documented here) 
\end{itemize}

\subsection*{Requirements}
\index[general]{Requirements }
\addcontentsline{toc}{subsection}{Requirements}

In addition, to the above assumptions, the following conditions or
restrictions apply: 

\begin{itemize}
\item Linux only -- tested only on Red Hat, but should work on other Linuxes  
\item The scripts handle only SCSI and IDE disks  
\item All partitions will be recreated, but only {\bf ext2},  {\bf ext3}, {\bf
   rfs} and {\bf swap} partitions will be reformatted.  Any other partitions such
   as Windows FAT partitions will  not be formatted by the scripts, but you can
   do it by hand  
\item You are using either {\bf lilo} or {\bf grub} as a boot  loader, and you
   know which one (not automatically detected)  
\item The partitioning and reformating scripts will *should* work with RAID 
   devices, but probably not with other ``complicated'' disk 
   partitioning/formating schemes. They also should work with  Reiser
   filesystems. Please check them carefully. You  will probably need to edit the
   scripts by hand to make them work.  
\item You will need mkisofs (might be part of cdrtools, but is a separate rpm 
   on my system); cdrecord or some other tool for burning the CDROM. 
   \end{itemize}

\subsection*{Directories}
\index[general]{Directories }
\addcontentsline{toc}{subsection}{Directories}

To build the Bacula Rescue CDROM, you will find the necessary scripts in {\bf
rescue/linux/cdrom} subdirectory of the Bacula source code. If you installed
the bacula-rescue rpm package the scripts will be found in the {\bf
/etc/bacula/rescue/cdrom} directory. 

\subsection*{Preparation for a Bare Metal Recovery}
\index[general]{Recovery!Preparation for a Bare Metal }
\index[general]{Preparation for a Bare Metal Recovery }
\addcontentsline{toc}{subsection}{Preparation for a Bare Metal Recovery}

Before you can do a Bare Metal recovery, you must create a Bacula Rescue
CDROM, which will contain everything you need to begin recovery. This assumes
that you will have your Directory and Storage daemon running on a different
machine. If you want to recover a machine where the Director and/or the
database were previously running things will be much more complicated. 

\subsection*{Creating a Bacula Rescue CDROM}
\index[general]{CDROM!Creating a Bacula Rescue }
\index[general]{Creating a Bacula Rescue CDROM }
\addcontentsline{toc}{subsection}{Creating a Bacula Rescue CDROM}

The primary goals of the Bacula rescue CD are: 

\begin{itemize}
\item NOT to be a general or universal recovery disk. 
\item to capture and setup a restore environment for a  single system running
   as a Client. 
\item to capture the current state of the hard disks on your system, so that
   they can be easily restored from pre-generated  scripts. Note, this is
   not done by any other rescue CDROM, as far as I am aware.
\item to create and save a statically linked copy of your  current Bacula FD. 
   Thus you need no packages or other software to be installed before using
   this CDROM and the Bacula File daemon on it.
\item to be relatively easy to create. In most cases  you simply type {\bf
   make all} in the {\bf rescue/linux/cdrom}  directory, then burn the ISO image
   created. In contrast,  if you have looked at  any of the documentation on how
   to remaster a CD or how to roll your  own, your head will spin (at least mine
   did). 
\item to be easy for you to add any additional files, binaries,  or libraries
   to the CD. 
\item to build and work on any (or almost any) Linux  flavor or release. 
\item you might ask why I don't use Knoppix or some other preprepared recovery
   disk, especially since Knoppix is very kind and provides the Bacula FD on
   their disk.  The answer is that: I am more comfortable having my Linux boot
   up in rescue mode rather than another flavor. In addition, the Bacula rescue
   CDROM contains a complete snapshot of your disk partitioning, which is not
   the case with any other rescue disk. If your harddisk dies, do you remember all
   the partitions you had and how big they are?  I don't, and without that information,
   you have little hope of reformatting your harddisk and rebuilding your system.
\end{itemize}

One of the main of the advantages of a Bacula Rescue CDROM is that it contains
a bootable copy of your system, so you should be familiar with it. 

You should probably make a new rescue CDROM each time you make any major
updates to your kernel, and every time you upgrade a major version of Bacula. 

The whole process with the exception of burning the CDROM is done with the
following commands: 

\footnotesize
\begin{verbatim}
(Build a working version of Bacula in the
 bacula-source directory)
cd <bacula-source>
./configure (your options)
make
cd <bacula-source>/rescue/linux/cdrom
su (become root)
make all
\end{verbatim}
\normalsize

For users of the bacula-rescue rpm the static bacula-fd has already been built
and placed in {\bf /etc/bacula/rescue/cdrom/bin/} along with a symbolic link
to your {\bf /etc/bacula/bacula-fd.conf} file. Rpm users only need to do the
second step: 

\footnotesize
\begin{verbatim}
cd /etc/bacula/rescue/cdrom
su (become root)
make all
\end{verbatim}
\normalsize

At this point, if the scripts are successful, they should have done the
following things: 

\begin{itemize}
\item Made a copy of your kernel and its essential files.  
\item Copied a number of binary files from your system.  
\item Copied all the necessary shared libraries to run the above  binary
   files.  
\item Made a statically-linked version of your File daemon and  copied it into
   the CDROM build area.  
\item Made an ISO image and left it in {\bf bootcd.iso} 
   \end{itemize}

Once this is accomplished, you need only burn it into a CDROM. This can be
done directly from the makefile with: 

\footnotesize
\begin{verbatim}
make burn
\end{verbatim}
\normalsize

However, you may need to modify the Makefile to properly specify your CD
burner as the detection process is complicated especially if you have two
CDROMs or do not have {\bf cdrecord} loaded on your system. Users of the
rescue rpm package should definitely examine the Makefile since it was
configured on the host used to produce the rpm package. If you find that the
{\bf make burn} does not work for you, try doing a: 

\footnotesize
\begin{verbatim}
make scan
\end{verbatim}
\normalsize

and use the output of that to modify the Makefile accordingly. 

The ``make all'' that you did above actually does the equivalent to the
following: 

\footnotesize
\begin{verbatim}
make kernel
make binaries
make bacula
make iso
\end{verbatim}
\normalsize

If you wish, you can modify what you put on the CDROM and redo any part of the
make that you wish. For example, if you want to add a new directory, you might
do the first three makes, then add a new directory to the CDROM, and finally
do a ``make iso''. Please see the README file in the {\bf rescue/linux/cdrom}
or {\bf /etc/bacula/rescue/cdrom}directory for instructions on changing the
contents of the CDROM. 

At the current time, the size of the CDROM is about 50MB (compressed to about
20MB), so there is quite a bit more room for additional program. Keep in mind
that when this CDROM is booted, *everything* is in memory, so the total size
cannot exceed your memory size, and even then you will need some reserve
memory for running programs, ... 
\label{twosystemcd}

\subsection*{Putting Two or More Systems on Your Rescue Disk}
\index[general]{Putting Two or More Systems on Your Rescue Disk }
\index[general]{Disk!Putting Two or More Systems on Your Rescue }
\addcontentsline{toc}{subsection}{Putting Two or More Systems on Your Rescue
Disk}

You can put multiple systems on the same rescue CD if you wish. This is
because the information that is specific to your OS will be stored in the {\bf
/bacula-hostname} directory, where {\bf hostname} is the name of the host on
which you are building the CD. Suppose for example, you have two systems. One
named {\bf client1} and one named {\bf client2}. Assume also that your CD
burner is on client1, and that is the machine we start on, and that we can ssh
into client2 and also client2's disks are mounted on client1. 

\footnotesize
\begin{verbatim}
ssh client2
cd <bacula-source>
./configure (your options)
make
cd rescue/linux/cdrom
su
(enter root password)
make bacula
exit
exit
\end{verbatim}
\normalsize

Again, for rpm package users the above command set would be: 

\footnotesize
\begin{verbatim}
ssh client2
cd /etc/bacula/rescue/cdrom
su
(enter root password)
make bacula
exit
exit
\end{verbatim}
\normalsize

Thus we have just built a Bacula rescue directory on client2. Now, on client1,
we copy the appropriate directory to two places (explained below), then build
an ISO and burn it: 

\footnotesize
\begin{verbatim}
cd <bacula-source>
./configure (your options)
make
cd rescue/linux/cdrom
su
(enter root password)
c=/mnt/client2/home/user/bacula/rescue/linux/cdrom
cp -a $c/roottree/bacula-client2 roottree
cp -a $c/roottree/bacula-client2 cdtree
make all
make burn
exit
\end{verbatim}
\normalsize

And with the rpm package: 

\footnotesize
\begin{verbatim}
cd /etc/bacula/rescue/cdrom
su
(enter root password)
c=/mnt/client2/etc/bacula/rescue/cdrom
cp -a $c/roottree/bacula-client2 roottree
cp -a $c/roottree/bacula-client2 cdtree
make all
make burn
exit
\end{verbatim}
\normalsize

In summary, with the above commands, we first build a Bacula directory on
client2 in roottree/bacula-client2, then we copied the bacula-client2
directory into the client1's roottree so it is available in memory after
booting, and we also copied it into the cdtree so it will also be on the CD as
a separate directory and thus can be read without booting the CDROM. Then we
made and burned the CDROM for client1, which of course, contains the client2
data. 
\label{restore}

\subsection*{Restoring a Client System}
\index[general]{Restoring a Client System }
\index[general]{System!Restoring a Client }
\addcontentsline{toc}{subsection}{Restoring a Client System}

Now, let's assume that your hard disk has just died and that you have replaced
it with an new identical drive. In addition, we assume that you have: 

\begin{enumerate}
\item A recent Bacula backup (Full plus Incrementals)  
\item A Bacula Rescue CDROM.  
\item Your Bacula Director, Catalog, and Storage daemon running  on another
   machine on your local network. 
   \end{enumerate}

This is a relatively simple case, and later in this chapter, as time permits,
we will discuss how you might recover from a situation where the machine that
crashes is your main Bacula server (i.e. has the Director, the Catalog, and
the Storage daemon). 

You will take the following steps to get your system back up and running: 

\begin{enumerate}
\item Boot with your Bacula Rescue CDROM.  
\item Start the Network (local network)  
\item Re-partition your hard disk(s) as it was before  
\item Re-format your partitions  
\item Restore the Bacula File daemon (static version)  
\item Perform a Bacula restore of all your files  
\item Re-install your boot loader  
\item Reboot 
   \end{enumerate}

Now for the details ... 

\subsection*{Boot with your Bacula Rescue CDROM}
\index[general]{CDROM!Boot with your Bacula Rescue }
\index[general]{Boot with your Bacula Rescue CDROM }
\addcontentsline{toc}{subsection}{Boot with your Bacula Rescue CDROM}

When the CDROM boots, you will be presented with a script that looks like: 

\footnotesize
\begin{verbatim}
 
      Welcome to the Bacula Rescue Disk 1.1.0
To proceed, press the <ENTER> key or type "linux <runlevel>"
 
   linux 1     -> shell
   linux 2     -> login  (default if ENTER pressed)
   linux 3     -> network started and login (network not working yet)
   linux debug -> print debug during boot then login
\end{verbatim}
\normalsize

Normally, at this point, you simply press ENTER. However, you may supply
options for the boot if you wish. 

Once it has booted, you will be requested to login something like: 

\footnotesize
\begin{verbatim}
Welcome to the Bacula Rescue CDROM
2.4.21-15.0.4.EL #1 Wed Aug 4 03:08:03 EDT 2004
Please login using root and your root password ...
RescueCD login:
\end{verbatim}
\normalsize

Note, you must enter the root password for the system on which you loaded the
kernel or on which you did the build of the CDROM. Once you are logged in,
your will be in the home directory for {\bf root}, and you can proceed to
examine your system. 

The complete Bacula rescue part of the CD will be in the directory: {\bf
/bacula-hostname}, where hostname is replaced by the name of the host machine
on which you did the build for the CDROM. This naming procedure allows you to
put multiple restore environments for each of your machines on a single CDROM
if you so wish to do. Please see the README document in the {\bf
rescue/linux/cdrom} directory for more information on adding to the CDROM. 

\paragraph*{Start the Network:}

At this point, you should bring up your network. Normally, this is quite
simple and requires just a few commands. Please cd into the /bacula-hostname
directory before continuing. To simplify your task, we have created a script
that should work in most cases by typing: 

\footnotesize
\begin{verbatim}
cd /bacula-hostname
./start_network
\end{verbatim}
\normalsize

You can test it by pinging another machine, or pinging your broken machine
machine from another machine. Do not proceed until your network is up. 

\paragraph*{Partition Your Hard Disk(s):}

Assuming that your hard disk crashed and needs repartitioning, proceed with: 

\footnotesize
\begin{verbatim}
./partition.hda
\end{verbatim}
\normalsize

If you have multiple disks, do the same for each of them. For SCSI disks, the
repartition script will be named: {\bf partition.sda}. If the script complains
about the disk being in use, simply go back and redo the {\bf df} command and
{\bf umount} commands until you no longer have your hard disk mounted. Note,
in many cases, if your hard disk was seriously damaged or a new one installed,
it will not automatically be mounted. If it is mounted, it is because the
emergency kernel found one or more possibly valid partitions. 

If for some reason this procedure does not work, you can use the information
in {\bf partition.hda} to re-partition your disks by hand using {\bf fdisk}. 

\paragraph*{Format Your Hard Disk(s):}

If you have repartitioned your hard disk, you must format it appropriately.
The formatting script will put back swap partitions, normal Unix partitions
(ext2) and journaled partitions (ext3) as well as Reiser partitions (rei). Do
so by entering for each disk: 

\footnotesize
\begin{verbatim}
./format.hda
\end{verbatim}
\normalsize

The format script will ask you if you want a block check done. We recommend to
answer yes, but realize that for very large disks this can take hours. 

\paragraph*{Mount the Newly Formatted Disks:}

Once the disks are partitioned and formatted, you can remount them with the
{\bf mount\_drives} script. All your drives must be mounted for Bacula to be
able to access them. Run the script as follows: 

\footnotesize
\begin{verbatim}
./mount_drives
df
\end{verbatim}
\normalsize

The {\bf df} command will tell you if the drives are mounted. If not, re-run
the script again. It isn't always easy to figure out and create the mount
points and the mounts in the proper order, so repeating the {\bf
./mount\_drives} command will not cause any harm and will most likely work the
second time. If not, correct it by hand before continuing. 

\paragraph*{Restore and Start the File Daemon:}

If you have booted with a Bacula Rescue CDROM, your statically linked Bacula
File daemon and the bacula-fd.conf file with be in the /bacula-hostname/bin
directory. Make sure {\bf bacula-fd} and {\bf bacula-fd.conf} are both there. 

Edit the Bacula configuration file, create the working/pid/subsys directory if
you haven't already done so above, and start Bacula. Before starting Bacula,
you will need to move it and bacula-fd.conf from /bacula-hostname/bin, to the
/mnt/disk/tmp directory so that it will be on your hard disk. Then start it
with the following command: 

\footnotesize
\begin{verbatim}
chroot /mnt/disk /tmp/bacula-fd -c /tmp/bacula-fd.conf
\end{verbatim}
\normalsize

The above command starts the Bacula File daemon with your the proper root disk
location (i.e. {\bf /mnt/disk/tmp}. If Bacula does not start correct the
problem and start it. You can check if it is running by entering: 

\footnotesize
\begin{verbatim}
ps fax
\end{verbatim}
\normalsize

You can kill Bacula by entering: 

\footnotesize
\begin{verbatim}
kill -TERM <pid>
\end{verbatim}
\normalsize

where {\bf pid} is the first number printed in front of the first occurrence
of {\bf bacula-fd} in the {\bf ps fax} command. 

Now, you should be able to use another computer with Bacula installed to check
the status by entering: 

\footnotesize
\begin{verbatim}
status client=xxxx
\end{verbatim}
\normalsize

into the Console program, where xxxx is the name of the client you are
restoring. 

One common problem is that your {\bf bacula-dir.conf} may contain machine
addresses that are not properly resolved on the stripped down system to be
restored because it is not running DNS. This is particularly true for the
address in the Storage resource of the Director, which may be very well
resolved on the Director's machine, but not on the machine being restored and
running the File daemon. In that case, be prepared to edit {\bf
bacula-dir.conf} to replace the name of the Storage daemon's domain name with
its IP address. 

\paragraph*{Restore Your Files:}

On the computer that is running the Director, you now run a {\bf restore}
command and select the files to be restored (normally everything), but before
starting the restore, there is one final change you must make using the {\bf
mod} option. You must change the {\bf Where} directory to be the root by using
the {\bf mod} option just before running the job and selecting {\bf Where}.
Set it to: 

\footnotesize
\begin{verbatim}
/
\end{verbatim}
\normalsize

then run the restore. 

You might be tempted to avoid using {\bf chroot} and running Bacula directly
and then using a {\bf Where} to specify a destination of {\bf /mnt/disk}. This
is possible, however, the current version of Bacula always restores files to
the new location, and thus any soft links that have been specified with
absolute paths will end up with {\bf /mnt/disk} prefixed to them. In general
this is not fatal to getting your system running, but be aware that you will
have to fix these links if you do not use {\bf chroot}. 

\paragraph*{Final Step:}

At this point, the restore should have finished with no errors, and all your
files will be restored. One last task remains and that is to write a new boot
sector so that your machine will boot. For {\bf lilo}, you enter the following
command: 

\footnotesize
\begin{verbatim}
./run_lilo
\end{verbatim}
\normalsize

If you are using grub instead of lilo, you must enter the following: 

\footnotesize
\begin{verbatim}
./run_grub
\end{verbatim}
\normalsize

Note, I've had quite a number of problems with {\bf grub} because it is rather
complicated and not designed to install easily under a simplified system. So,
if you experience errors or end up unexpectedly in a {\bf chroot} shell,
simply exit back to the normal shell and type in the appropriate commands from
the {\bf run\_grub} script by hand until you get it to install. When you run
the run\_grub script, it will print the commands that you should manually
enter if that is necessary. 

\paragraph*{Reboot:}

First unmount all your hard disks, otherwise they will not be cleanly
shutdown, then reboot your machine by entering {\bf exit} until you get to the
main prompt then enter {\bf ctl-d}. Once back to the main CDROM prompt, you
will need to turn the power off then back on to your machine to get it to
reboot. 

If everything went well, you should now be back up and running. If not,
re-insert the emergency boot CDROM, boot, and figure out what is wrong. 
\label{server}

\subsection*{Restoring a Server}
\index[general]{Restoring a Server }
\index[general]{Server!Restoring a }
\addcontentsline{toc}{subsection}{Restoring a Server}

Above, we considered how to recover a client machine where a valid Bacula
server was running on another machine. However, what happens if your server
goes down and you no longer have a running Director, Catalog, or Storage
daemon? There are several solutions: 

\begin{enumerate}
\item Bring up static versions of your Director, Catalog, and Storage  daemon.

\item Move your server to another machine. 
   \end{enumerate}

The first option, is very difficult because it requires you to have created a
static version of the Director and the Storage daemon as well as the Catalog.
If the Catalog uses MySQL or PostgreSQL, this may or may not be possible. In
addition, to loading all these programs on a bare system (quite possible), you
will need to make sure you have a valid driver for your tape drive. 

The second suggestion is probably a much simpler solution, and one I have done
myself. To do so, you might want to consider the following steps: 

\begin{itemize}
\item If you are using MySQL or PostgreSQL, configure, build and install it
   from  source (or user rpms) on your new system.  
\item Load the Bacula source code onto your new system, configure,  install
   it, and create the Bacula database.  
\item If you have a valid saved Bootstrap file as created for your  damaged
   machine with WriteBootstrap, use it to restore the  files to the damaged
   machine, where you have loaded a static Bacula  File daemon using the Bacula
Rescue disk). This is done by  using the restore command and at the yes/mod/no
prompt,  selecting {\bf mod} then specifying the path to the bootstrap  file. 
\item If you have the Bootstrap file, you should now be back up and  running,
   if you do not have a Bootstrap file, continue with the  suggestions below.  
\item Using {\bf bscan} scan the last set of backup tapes into your  MySQL,
   PostgreSQL or SQLite database.  
\item Start Bacula, and using the Console {\bf restore} command,  restore the
   last valid copy of the Bacula database and the  the Bacula configuration
   files.  
\item Move the database to the correct location. 
\item Start the database, and restart Bacula. Then use  the Console {\bf
   restore} command, restore all the files  on the damaged machine, where you
   have loaded a Bacula File  daemon using the Bacula Rescue disk. 
\end{itemize}

\label{problems2}

\subsection*{Linux Problems or Bugs}
\index[general]{Bugs!Linux Problems or }
\index[general]{Linux Problems or Bugs }
\addcontentsline{toc}{subsection}{Linux Problems or Bugs}

Since every flavor and every release of Linux is different, there are likely
to be some small difficulties with the scripts, so please be prepared to edit
them in a minimal environment. A rudimentary knowledge of {\bf vi} is very
useful. Also, these scripts do not do everything. You will need to reformat
Windows partitions by hand, for example. 

Getting the boot loader back can be a problem if you are using {\bf grub}
because it is so complicated. If all else fails, reboot your system from your
floppy but using the restored disk image, then proceed to a reinstallation of
grub (looking at the run-grub script can help). By contrast, lilo is a piece
of cake. 
\label{FreeBSD1}

\subsection*{FreeBSD Bare Metal Recovery}
\index[general]{Recovery!FreeBSD Bare Metal }
\index[general]{FreeBSD Bare Metal Recovery }
\addcontentsline{toc}{subsection}{FreeBSD Bare Metal Recovery}

The same basic techniques described above also apply to FreeBSD. Although we
don't yet have a fully automated procedure, Alex Torres Molina has provided us
with the following instructions with a few additions from Jesse Guardiani and
Dan Languille: 

\begin{enumerate}
\item Boot with the FreeBSD installation disk 
\item Go to Custom, Partition and create your slices and go to Label and 
   create the particions that you want. Apply changes. 
\item Go to Fixit to start a emergency console. 
\item Create devs ad0 .. .. if don't exist under /mnt2/dev (in my  situation)
   with MAKEDEV. The device or devices you  create depend on what hard drives you
   have. ad0 is your  first ATA drive. da0 would by your first SCSI drive.  Under
OS version 5 and greater, your device files are  most likely automatically
created for you. 
\item mkdir /mnt/disk
   this is the root of the new disk 
\item mount /mnt2/dev/ad0s1a /mnt/disk
   mount /mnt2/dev/ad0s1c /mnt/disk/var
   mount /mnt2/dev/ad0s1d /mnt/disk/usr
.....
The same hard drive isssues as above apply here too.  Note, under OS version 5
or higher, your disk devices may  be in /dev not /mnt2/dev. 
\item Network configuraion (ifconfig xl0 ip/mask + route add default 
   ip-gateway) 
\item mkdir /mnt/disk/tmp 
\item cd /mnt/disk/tmp 
\item Copy bacula-fd and bacula-fd.conf to this path 
\item If you need to use sftp to copy files then you must do this:
   ln -s /mnt2/usr/bin /usr/bin 
\item chmod u+x bacula-fd 
\item Modify bacula-fd.conf to fit this machine 
\item Copy /bin/sh to /mnt/disk, neccesary for chroot 
\item Don't forget to put your bacula-dir's IP address and domain  name in
   /mnt/disk/etc/hosts if it's not on a public net.  Otherwise the FD on the
   machine you are restoring to  won't be able to contact the SD and DIR on the
remote machine. 
\item mkdir -p /mnt/disk/var/db/bacula 
\item chroot /mnt/disk /tmp/bacula-fd -c /tmp/bacula-fd.conf
   to start bacula-fd 
\item Now you can go to bacula-dir and restore the job with the entire 
   contents of the broken server. 
\item You must create /proc 
   \end{enumerate}

\label{solaris}

\subsection*{Solaris Bare Metal Recovery}
\index[general]{Solaris Bare Metal Recovery }
\index[general]{Recovery!Solaris Bare Metal }
\addcontentsline{toc}{subsection}{Solaris Bare Metal Recovery}

The same basic techniques described above apply to Solaris: 

\begin{itemize}
\item the same restrictions as those given for Linux apply  
\item you will need to create a Bacula Rescue disk 
   \end{itemize}

However, during the recovery phase, the boot and disk preparation procedures
are different: 

\begin{itemize}
\item there is no need to create an emergency boot disk  since it is an
   integrated part of the Solaris boot.  
\item you must partition and format your hard disk by hand  following manual
   procedures as described in W. Curtis Preston's  book ``Unix Backup \&
   Recovery'' 
\end{itemize}

Once the disk is partitioned, formatted and mounted, you can continue with
bringing up the network and reloading Bacula. 

\subsection*{Preparing Solaris Before a Disaster}
\index[general]{Preparing Solaris Before a Disaster }
\index[general]{Disaster!Preparing Solaris Before a }
\addcontentsline{toc}{subsection}{Preparing Solaris Before a Disaster}

As mentioned above, before a disaster strikes, you should prepare the
information needed in the case of problems. To do so, in the {\bf
rescue/solaris} subdirectory enter: 

\footnotesize
\begin{verbatim}
su
./getdiskinfo
./make_rescue_disk
\end{verbatim}
\normalsize

The {\bf getdiskinfo} script will, as in the case of Linux described above,
create a subdirectory {\bf diskinfo} containing the output from several system
utilities. In addition, it will contain the output from the {\bf SysAudit}
program as described in Curtis Preston's book. This file {\bf
diskinfo/sysaudit.bsi} will contain the disk partitioning information that
will allow you to manually follow the procedures in the ``Unix Backup \&
Recovery'' book to repartition and format your hard disk. In addition, the
{\bf getdiskinfo} script will create a {\bf start\_network} script. 

Once you have your your disks repartitioned and formatted, do the following: 

\begin{itemize}
\item Start Your Network with the {\bf start\_network} script  
\item Restore the Bacula File daemon as documented above  
\item Perform a Bacula restore of all your files using the same  commands as
   described above for Linux  
\item Re-install your boot loader using the instructions outlined  in the
   ``Unix Backup \& Recovery'' book  using installboot 
   \end{itemize}

\label{genbugs}

\subsection*{Bugs and Other Considerations}
\index[general]{Considerations!Bugs and Other }
\index[general]{Bugs and Other Considerations }
\addcontentsline{toc}{subsection}{Bugs and Other Considerations}

\paragraph*{Directory Modification and Access Times are Modified on pre-1.30
Baculas :}

When a pre-1.30 version of Bacula restores a directory, it first must create
the directory, then it populates the directory with its files and
subdirectories. The act of creating the files and subdirectories updates both
the modification and access times associated with the directory itself. As a
consequence, all modification and access times of all directories will be
updated to the time of the restore. 

This has been corrected in Bacula version 1.30 and later. The directory
modification and access times is reset to the value saved in the backup after
all the files and subdirectories have been restored. This has been tested and
verified on normal restore operations, but not verified during a bare metal
recovery. 

\paragraph*{Strange Bootstrap Files:}

If any of you look closely at the bootstrap file that is produced and used for
the restore (I sure do), you will probably notice that the FileIndex item does
not include all the files saved to the tape. This is because in some instances
there are duplicates (especially in the case of an Incremental save), and in
such circumstances, {\bf Bacula} restores only the last of multiple copies of
a file or directory. 
\label{Win3233}

\subsection*{Disaster Recovery of Win32 Systems}
\index[general]{Systems!Disaster Recovery of Win32 }
\index[general]{Disaster Recovery of Win32 Systems }
\addcontentsline{toc}{subsection}{Disaster Recovery of Win32 Systems}

Due to open system files, and registry problems, Bacula cannot save and
restore a complete Win2K/XP/NT environment. 

A suggestion by Damian Coutts using Microsoft's NTBackup utility in
conjunction with Bacula should permit a Full bare metal restore of Win2K/XP
(and possibly NT systems). His suggestion is to do an NTBackup of the critical
system state prior to running a Bacula backup with the following command: 

\footnotesize
\begin{verbatim}
ntbackup backup systemstate /F c:\systemstate.bkf
\end{verbatim}
\normalsize

The {\bf backup} is the command, the {\bf systemstate} says to backup only the
system state and not all the user files, and the {\bf /F
c:\textbackslash{}systemstate.bkf} specifies where to write the state file.
this file must then be saved and restored by Bacula. 

To restore the system state, you first reload a base operating system, then
you would use Bacula to restore all the users files and to recover the {\bf
c:\textbackslash{}systemstate.bkf} file, and finally, run {\bf NTBackup} and
{\bf catalogue} the system statefile, and then select it for restore. The
documentation says you can't run a command line restore of the systemstate. 

This procedure has been confirmed to work by Ludovic Strappazon -- many
thanks! 

A new tool is provided in the form of a bacula plugin for the BartPE rescue
CD. BartPE is a self-contained WindowsXP boot CD which you can make using the
PeBuilder tools available at 
\elink{http://www.nu2.nu/pebuilder/}{http://www.nu2.nu/pebuilder/} and a valid
Windows XP SP1 CDROM. The plugin is provided as a zip archive. Unzip the file
and copy the bacula directory into the plugin directory of your BartPE
installation. Edit the configuration files to suit your installation and build
your CD according to the instructions at Bart's site. This will permit you to
boot from the cd, configure and start networking, start the bacula file client
and access your director with the console program. The programs menu on the
booted CD contains entries to install the file client service, start the file
client service, and start the WX-Console. You can also open a command line
window and CD Programs\textbackslash{}Bacula and run the command line console
bconsole. 

\subsection*{Resetting Directory and File Ownership and Permissions on Win32
Systems}
\index[general]{Systems!Resetting Directory and File Ownership and Permissions
on Win32 }
\index[general]{Resetting Directory and File Ownership and Permissions on
Win32 Systems }
\addcontentsline{toc}{subsection}{Resetting Directory and File Ownership and
Permissions on Win32 Systems}

Bacula versions after 1.31 should properly restore ownership and permissions
on all WinNT/XP/2K systems. If you do experience problems, generally in
restores to alternate directories because higher level directories were not
backed up by Bacula, you can correct any problems with the {\bf SetACL}
available under the GPL license at: 
\elink{http://sourceforge.net/projects/setacl/}{http://sourceforge.net/project%
s/setacl/}. 

\subsection*{Alternate Disaster Recovery Suggestion for Win32 Systems}
\index[general]{Systems!Alternate Disaster Recovery Suggestion for Win32 }
\index[general]{Alternate Disaster Recovery Suggestion for Win32 Systems }
\addcontentsline{toc}{subsection}{Alternate Disaster Recovery Suggestion for
Win32 Systems}

Ludovic Strappazon has suggested an interesting way to backup and restore
complete Win32 partitions. Simply boot your Win32 system with a Linux Rescue
disk as described above for Linux, install a statically linked Bacula, and
backup any of the raw partitions you want. Then to restore the system, you
simply restore the raw partition or partitions. Here is the email that Ludovic
recently sent on that subject: 

\footnotesize
\begin{verbatim}
I've just finished testing my brand new cd LFS/Bacula
with a raw Bacula backup and restore of my portable.
I can't resist sending you the results: look at the rates !!!
hunt-dir: Start Backup JobId 100, Job=HuntBackup.2003-04-17_12.58.26
hunt-dir: Bacula 1.30 (14Apr03): 17-Apr-2003 13:14
JobId:                  100
Job:                    HuntBackup.2003-04-17_12.58.26
FileSet:                RawPartition
Backup Level:           Full
Client:                 sauvegarde-fd
Start time:             17-Apr-2003 12:58
End time:               17-Apr-2003 13:14
Files Written:          1
Bytes Written:          10,058,586,272
Rate:                   10734.9 KB/s
Software Compression:   None
Volume names(s):        000103
Volume Session Id:      2
Volume Session Time:    1050576790
Last Volume Bytes:      10,080,883,520
FD termination status:  OK
SD termination status:  OK
Termination:            Backup OK
hunt-dir: Begin pruning Jobs.
hunt-dir: No Jobs found to prune.
hunt-dir: Begin pruning Files.
hunt-dir: No Files found to prune.
hunt-dir: End auto prune.
hunt-dir: Start Restore Job RestoreFilesHunt.2003-04-17_13.21.44
hunt-sd: Forward spacing to file 1.
hunt-dir: Bacula 1.30 (14Apr03): 17-Apr-2003 13:54
JobId:                  101
Job:                    RestoreFilesHunt.2003-04-17_13.21.44
Client:                 sauvegarde-fd
Start time:             17-Apr-2003 13:21
End time:               17-Apr-2003 13:54
Files Restored:         1
Bytes Restored:         10,056,130,560
Rate:                   5073.7 KB/s
FD termination status:  OK
Termination:            Restore OK
hunt-dir: Begin pruning Jobs.
hunt-dir: No Jobs found to prune.
hunt-dir: Begin pruning Files.
hunt-dir: No Files found to prune.
hunt-dir: End auto prune.
\end{verbatim}
\normalsize

\label{running}

\subsection*{Restoring to a Running System}
\index[general]{System!Restoring to a Running }
\index[general]{Restoring to a Running System }
\addcontentsline{toc}{subsection}{Restoring to a Running System}

If for some reason you want to do a Full restore to a system that has a
working kernel, you will need to take care not to overwrite the following
files: 

\footnotesize
\begin{verbatim}
/etc/grub.conf
/etc/X11/Conf
/etc/fstab
/etc/mtab
/lib/modules
/usr/modules
/usr/X11R6
/etc/modules.conf
\end{verbatim}
\normalsize

\label{Resources}

\subsection*{Additional Resources}
\index[general]{Additional Resources }
\index[general]{Resources!Additional }
\addcontentsline{toc}{subsection}{Additional Resources}

Many thanks to Charles Curley who wrote 
\elink{ Linux Complete Backup and Recovery
HOWTO}
{http://www.tldp.org/HOWTO/Linux-Complete-Backup-and-Recovery-HOWTO/index.html%
} for the 
\elink{The Linux Documentation Project}{http://www.tldp.org/}. This is an
excellent document on how to do Bare Metal Recovery on Linux systems, and it
was this document that made me realize that Bacula could do the same thing. 

You can find quite a few additional resources, both commercial and free at 
\elink{Storage Mountain}{http://www.backupcentral.com}, formerly known as
Backup Central. 

And finally, the O'Reilly book, ``Unix Backup \& Recovery'' by W. Curtis
Preston covers virtually every backup and recovery topic including bare metal
recovery for a large range of Unix systems. 
