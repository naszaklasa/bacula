%%
%%

\section*{Installing and Configuring PostgreSQL}
\label{_ChapterStart10}
\index[general]{PostgreSQL!Installing and Configuring }
\index[general]{Installing and Configuring PostgreSQL }
\addcontentsline{toc}{section}{Installing and Configuring PostgreSQL}

\subsection*{Installing and Configuring PostgreSQL -- Phase I}
\index[general]{Installing and Configuring PostgreSQL -- Phase I }
\index[general]{Phase I!Installing and Configuring PostgreSQL -- }
\addcontentsline{toc}{subsection}{Installing and Configuring PostgreSQL --
Phase I}

If you use the {\bf ./configure \verb{--{with-postgresql=PostgreSQL-Directory}
statement for configuring {\bf Bacula}, you will need PostgreSQL version 7.3
or later installed. NOTE! PostgreSQL versions earlier than 7.3 do not work
with Bacula. If PostgreSQL is installed in the standard system location, you
need only enter {\bf \verb{--{with-postgresql} since the configure program will
search all the standard locations. If you install PostgreSQL in your home
directory or some other non-standard directory, you will need to provide the
full path with the {\bf \verb{--{with-postgresql} option. 

Installing and configuring PostgreSQL is not difficult but can be confusing
the first time. If you prefer, you may want to use a package provided by your
chosen operating system. Binary packages are available on most PostgreSQL
mirrors. 

If you prefer to install from source, we recommend following the instructions
found in the 
\elink{PostgreSQL documentation}{http://www.postgresql.org/docs/}. 

If you are using FreeBSD, 
\elink{this FreeBSD Diary article}{http://www.freebsddiary.org/postgresql.php}
will be useful. Even if you are not using FreeBSD, the article will contain
useful configuration and setup information. 

After installing PostgreSQL, you should return to completing the installation
of {\bf Bacula}. Later, after Bacula is installed, come back to this chapter
to complete the installation. Please note, the installation files used in the
second phase of the PostgreSQL installation are created during the Bacula
Installation. 
\label{PostgreSQL_phase2}

\subsection*{Installing and Configuring PostgreSQL -- Phase II}
\index[general]{Phase II!Installing and Configuring PostgreSQL -- }
\index[general]{Installing and Configuring PostgreSQL -- Phase II }
\addcontentsline{toc}{subsection}{Installing and Configuring PostgreSQL --
Phase II}

At this point, you should have built and installed PostgreSQL, or already have
a running PostgreSQL, and you should have configured, built and installed {\bf
Bacula}. If not, please complete these items before proceeding. 

Please note that the {\bf ./configure} used to build {\bf Bacula} will need to
include {\bf \verb{--{with-postgresql=PostgreSQL-directory}, where {\bf
PostgreSQL-directory} is the directory name that you specified on the
./configure command for configuring PostgreSQL (if you didn't specify a
directory or PostgreSQL is installed in a default location, you do not need to
specify the directory). This is needed so that Bacula can find the necessary
include headers and library files for interfacing to PostgreSQL. 

{\bf Bacula} will install scripts for manipulating the database (create,
delete, make tables etc) into the main installation directory. These files
will be of the form *\_bacula\_* (e.g. create\_bacula\_database). These files
are also available in the \lt{}bacula-src\gt{}/src/cats directory after
running ./configure. If you inspect create\_bacula\_database, you will see
that it calls create\_postgresql\_database. The *\_bacula\_* files are
provided for convenience. It doesn't matter what database you have chosen;
create\_bacula\_database will always create your database. 

Now you will create the Bacula PostgreSQL database and the tables that Bacula
uses. These instructions assume that you already have PostgreSQL running. You
will need to perform these steps as a user that is able to create new
databases. This can be the PostgreSQL user (on most systems, this is the pgsql
user). 

\begin{enumerate}
\item cd \lt{}install-directory\gt{}

   This directory contains the Bacula catalog  interface routines.  

\item ./create\_bacula\_database

   This script creates the PostgreSQL {\bf bacula} database.  

\item ./make\_bacula\_tables

   This script creates the PostgreSQL tables used by {\bf Bacula}.  
\item ./grant\_bacula\_privileges

   This script creates the database user {\bf bacula}  with restricted access
rights. You may  want to modify it to suit your situation. Please note that 
this database is not password protected.  

\end{enumerate}

Each of the three scripts (create\_bacula\_database, make\_bacula\_tables, and
grant\_bacula\_privileges) allows the addition of a command line argument.
This can be useful for specifying the user name. For example, you might need
to add {\bf -h hostname} to the command line to specify a remote database
server. 

To take a closer look at the access privileges that you have setup with the
above, you can do: 

\footnotesize
\begin{verbatim}
PostgreSQL-directory/bin/psql --command \\dp bacula
\end{verbatim}
\normalsize

\subsection*{Re-initializing the Catalog Database}
\index[general]{Database!Re-initializing the Catalog }
\index[general]{Re-initializing the Catalog Database }
\addcontentsline{toc}{subsection}{Re-initializing the Catalog Database}

After you have done some initial testing with {\bf Bacula}, you will probably
want to re-initialize the catalog database and throw away all the test Jobs
that you ran. To do so, you can do the following: 

\footnotesize
\begin{verbatim}
  cd <install-directory>
  ./drop_bacula_tables
  ./make_bacula_tables
  ./grant_bacula_privileges
\end{verbatim}
\normalsize

Please note that all information in the database will be lost and you will be
starting from scratch. If you have written on any Volumes, you must write and
end of file mark on the volume so that Bacula can reuse it. Do so with: 

\footnotesize
\begin{verbatim}
   (stop Bacula or unmount the drive)
   mt -f /dev/nst0 rewind
   mt -f /dev/nst0 weof
\end{verbatim}
\normalsize

Where you should replace {\bf /dev/nst0} with the appropriate tape drive
device name for your machine. 

\subsection*{Converting from MySQL to PostgreSQL}
\index[general]{PostgreSQL!Converting from MySQL to }
\index[general]{Converting from MySQL to PostgreSQL }
\addcontentsline{toc}{subsection}{Converting from MySQL to PostgreSQL}

The conversion procedure presented here was worked out by Norm Dressler
\lt{}ndressler at dinmar dot com\gt{} 

This process was tested using the following software versions: 

\begin{itemize}
\item Linux Mandrake 10/Kernel 2.4.22-10 SMP 
\item Mysql Ver 12.21 Distrib 4.0.15, for mandrake-linux-gnu (i586) 
\item PostgreSQL 7.3.4 
\item Bacula 1.34.5 
   \end{itemize}

WARNING: Always as a precaution, take a complete backup of your databases
before proceeding with this process! 

\begin{enumerate}
\item Shutdown bacula (cd /etc/bacula;./bacula stop)  
\item Run the following command to dump your Mysql database:  

   \footnotesize
\begin{verbatim}
       mysqldump -f -t -n >bacula-backup.dmp>
    
\end{verbatim}
\normalsize

\item Make a backup of your /etc/bacula directory (but leave the  original in
   place).  
\item Go to your Bacula source directory and rebuild it to include  PostgreSQL
   support rather then Mysql support. Check the  config.log file for your
   original configure command and replace  enable-mysql with enable-postgresql.  
\item Recompile Bacula with a make and if everything compiles  completely,
   perform a make install.  
\item Shutdown Mysql. 
\item Start PostgreSQL on your system.  
\item Create a bacula user in Postgres with createuser command.  Depending on
   your Postgres install, you may have to SU to the  user who has privileges to
   create a user.  
\item Verify your pg\_hba.conf file contains sufficient permissions to  allow
   bacula to access the server. Mine has the following since  it's on a secure
   network:  

\footnotesize
\begin{verbatim}
local all all trust
                
host all all 127.0.0.1 255.255.255.255 trust
                
NOTE: you should restart your postgres server if you
      made changes
      
\end{verbatim}
\normalsize

\item Change into the /etc/bacula directory and prepare the database  and
   tables with the following commands:  

\footnotesize
\begin{verbatim}
./create_postgresql_database
                                
./make_postgresql_tables
                                
./grant_postgresql_privileges
       
\end{verbatim}
\normalsize

\item Verify you have access to the database:  

   \footnotesize
\begin{verbatim}
  
psql -Ubacula bacula
      
\end{verbatim}
\normalsize

You should not get any errors.  
\item Load your database from the Mysql database dump with:  

   \footnotesize
\begin{verbatim}
psql -Ubacula bacula <bacula-backup.dmp>
      
\end{verbatim}
\normalsize

\item Reseqence your tables with the following commands:  

   \footnotesize
\begin{verbatim}
psql -Ubacula bacula
                
SELECT SETVAL('basefiles_baseid_seq', (SELECT
MAX(baseid) FROM basefiles));
SELECT SETVAL('client_clientid_seq', (SELECT
MAX(clientid) FROM client));
SELECT SETVAL('file_fileid_seq', (SELECT MAX(fileid)
FROM file));
SELECT SETVAL('filename_filenameid_seq', (SELECT
MAX(filenameid) FROM filename));
                
SELECT SETVAL('fileset_filesetid_seq', (SELECT
MAX(filesetid) FROM fileset));
                
SELECT SETVAL('job_jobid_seq', (SELECT MAX(jobid) FROM job));
SELECT SETVAL('jobmedia_jobmediaid_seq', (SELECT
MAX(jobmediaid) FROM jobmedia));
SELECT SETVAL('media_mediaid_seq', (SELECT MAX(mediaid) FROM media));
SELECT SETVAL('path_pathid_seq', (SELECT MAX(pathid) FROM path));
                
SELECT SETVAL('pool_poolid_seq', (SELECT MAX(poolid) FROM pool));
       
\end{verbatim}
\normalsize

\item At this point, start up Bacula, verify your volume library and  perform
   a test backup to make sure everything is working  properly. 
   \end{enumerate}

\subsection*{Credits}
\index[general]{Credits }
\addcontentsline{toc}{subsection}{Credits}

Many thanks to Dan Languille for writing the PostgreSQL driver. This will
surely become the most popular database that Bacula supports. 
