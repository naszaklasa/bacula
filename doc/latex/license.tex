%%
%%

\section*{Bacula Copyright, Trademark, and Licenses}
\label{_ChapterStart29}
\index[general]{Licenses!Bacula Copyright Trademark and }
\index[general]{Bacula Copyright, Trademark, and Licenses }
\addcontentsline{toc}{section}{Bacula Copyright, Trademark, and Licenses}

There are a number of different licenses that are used in Bacula. 

\subsection*{GPL}
\index[general]{GPL }
\addcontentsline{toc}{subsection}{GPL}

The vast bulk of the code is released under a modified version of the 
\ilink{GNU General Public License version 2.}{_ChapterStart20} The
modifications (actually additions) are described in the source file LICENSE,
and their purpose is not to alter the essential qualities of the GPL but to
permit more freedom in linking certain third party software supposedly non-GPL
compatable, provide termination for Patent (and IP) actions, clarify
contributors IP and Copyright claims and non-infringment intentions. The
details and governing text are in the file LICENSE in the main source
directory. 

Most of this code is copyrighted: Copyright \copyright 2000-2004 Kern Sibbald and
John Walker. or Copyright \copyright 2000-2005 Kern Sibbald 

Portions may be copyrighted by other people (ATT, the Free Software
Foundation, ...). Generally these portions are released under a
non-modified GPL 2 license.

\subsection*{LGPL}
\index[general]{LGPL }
\addcontentsline{toc}{subsection}{LGPL}

Some of the Bacula library source code is released under the 
\ilink{GNU Lesser General Public License.}{_ChapterStart49} This
permits third parties to use these parts of our code in their proprietary
programs to interface to Bacula. 

\subsection*{Public Domain}
\index[general]{Domain!Public }
\index[general]{Public Domain }
\addcontentsline{toc}{subsection}{Public Domain}

Some of the Bacula code has been released to the public domain. E.g. md5.c,
SQLite. 

\subsection*{Trademark}
\index[general]{Trademark }
\addcontentsline{toc}{subsection}{Trademark}

Bacula\raisebox{.6ex}{\textsuperscript{\textregistered}}is a registered
trademark of Kern Sibbald and John Walker. 

We have trademarked the Bacula name to ensure that any variant of Bacula will
be exactly compatible with the program that we have released. The use of the
name Bacula is restricted to software systems that agree exactly with the
program presented here. 

\subsection*{Disclaimer}
\index[general]{Disclaimer }
\addcontentsline{toc}{subsection}{Disclaimer}

NO WARRANTY 

BECAUSE THE PROGRAM IS LICENSED FREE OF CHARGE, THERE IS NO WARRANTY FOR THE
PROGRAM, TO THE EXTENT PERMITTED BY APPLICABLE LAW. EXCEPT WHEN OTHERWISE
STATED IN WRITING THE COPYRIGHT HOLDERS AND/OR OTHER PARTIES PROVIDE THE
PROGRAM ``AS IS'' WITHOUT WARRANTY OF ANY KIND, EITHER EXPRESSED OR IMPLIED,
INCLUDING, BUT NOT LIMITED TO, THE IMPLIED WARRANTIES OF MERCHANTABILITY AND
FITNESS FOR A PARTICULAR PURPOSE. THE ENTIRE RISK AS TO THE QUALITY AND
PERFORMANCE OF THE PROGRAM IS WITH YOU. SHOULD THE PROGRAM PROVE DEFECTIVE,
YOU ASSUME THE COST OF ALL NECESSARY SERVICING, REPAIR OR CORRECTION. 

IN NO EVENT UNLESS REQUIRED BY APPLICABLE LAW OR AGREED TO IN WRITING WILL ANY
COPYRIGHT HOLDER, OR ANY OTHER PARTY WHO MAY MODIFY AND/OR REDISTRIBUTE THE
PROGRAM AS PERMITTED ABOVE, BE LIABLE TO YOU FOR DAMAGES, INCLUDING ANY
GENERAL, SPECIAL, INCIDENTAL OR CONSEQUENTIAL DAMAGES ARISING OUT OF THE USE
OR INABILITY TO USE THE PROGRAM (INCLUDING BUT NOT LIMITED TO LOSS OF DATA OR
DATA BEING RENDERED INACCURATE OR LOSSES SUSTAINED BY YOU OR THIRD PARTIES OR
A FAILURE OF THE PROGRAM TO OPERATE WITH ANY OTHER PROGRAMS), EVEN IF SUCH
HOLDER OR OTHER PARTY HAS BEEN ADVISED OF THE POSSIBILITY OF SUCH DAMAGES. 
