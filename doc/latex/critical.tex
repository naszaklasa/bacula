%%
%%

\section*{Critical Items to Implement Before Going Production}
\label{_ChapterStart32}
\index[general]{Production!Critical Items to Implement Before Going }
\index[general]{Critical Items to Implement Before Going Production }
\addcontentsline{toc}{section}{Critical Items to Implement Before Going
Production}

\subsection*{General}
\index[general]{General }
\addcontentsline{toc}{subsection}{General}

We recommend you take your time before implementing a Bacula backup system
since Bacula is a rather complex program, and if you make a mistake, you may
suddenly find that you cannot restore the your files in case of a disaster.
This is especially true if you have not previously used a major backup
product. 

If you follow the instructions in this chapter, you will have covered most of
the major problems that can occur. It goes without saying that if ever you
find that we have left out an important point, please point it out to us, so
that we can document it to the benefit of everyone. 

\label{Critical}
\subsection*{Critical Items}
\index[general]{Critical Items }
\index[general]{Items!Critical }
\addcontentsline{toc}{subsection}{Critical Items}

The following assumes that you have installed Bacula, you more or less
understand it, you have at least worked through the tutorial or have
equivalent experience, and that you have setup a basic production
configuration. If you haven't done the above, please do so then come back
here. The following is a sort of checklist that points you elsewhere in the
manual with perhaps a brief explaination of why you should do it. The order is
more or less the order you would use in setting up a production system (if you
already are in production, use the checklist anyway). 

\begin{itemize}
\item Test your tape drive with compatibility with Bacula by using the  test
   command in the \ilink{btape}{btape} program. 
\item Better than doing the above is to walk through the nine steps in the  
   \ilink{Tape Testing}{_ChapterStart27} chapter of the manual. It 
   may take you a bit of time, but it will eliminate surprises. 
\item Make sure that /lib/tls is disabled. Bacula does not work with this 
   library. See the second point under 
   \ilink{ Supported Operating Systems.}{SupportedOSes} 
\item Do at least one restore of files. If you backup both Unix and Win32
   systems,  restore files from each system type. The 
   \ilink{Restoring Files}{_ChapterStart13} chapter shows you how. 
\item Write a bootstrap file to a separate system for each backup job.  The
   Write Bootstrap directive is described in the  
   \ilink{Director Configuration}{writebootstrap}  chapter of the
   manual, and more details are available in the  
   \ilink{Bootstrap File}{_ChapterStart43} chapter. Also, the default
   bacula-dir.conf comes with a Write Bootstrap directive defined. This  allows
   you to recover the state of your system as of the last backup.  
\item Backup your catalog. An example of this is found in the default 
   bacula-dir.conf file. The backup script is installed by default and  should
   handle any database, though you may want to make your own  local
   modifications.  
\item Write a bootstrap file for the catalog. An example of this is found in
   the default bacula-dir.conf file. This will allow you to quickly restore your
   catalog in the event it is wiped out -- otherwise it  is many excruciating
   hours of work.  
\item Make a Bacula Rescue CDROM! See the 
   \ilink{Disaster Recovery Using a Bacula Rescue
   CDROM}{_ChapterStart38} chapter. It is trivial to  make such a CDROM,
   and it can make system recovery in the event of  a lost hard disk infinitely
   easier. 
\end{itemize}

\subsection*{Recommended Items}
\index[general]{Items!Recommended }
\index[general]{Recommended Items }
\addcontentsline{toc}{subsection}{Recommended Items}

Although these items may not be critical, they are recommended and will help
you avoid problems. 

\begin{itemize}
\item Read the \ilink{Quick Start Guide to Bacula}{_ChapterStart37} 
\item After installing and experimenting with Bacula, read and work carefully 
   through the examples in the 
   \ilink{Tutorial}{_ChapterStart1} chapter  of this manual. 
\item Learn what each of the \ilink{Bacula Utility Programs}{_ChapterStart9}  does. 
\item Set up reasonable retention periods so that your catalog does not  grow
   to be too big. See the following three chapters:\\
   \ilink{Recycling your Volumes}{_ChapterStart22},\\
   \ilink{Basic Volume Management}{_ChapterStart39},\\
   \ilink{Using Pools to Manage Volumes}{_ChapterStart11}. 
\item Perform a bare metal recovery using the Bacula Rescue CDROM.  See the 
   \ilink{Disaster Recovery Using a Bacula Rescue CDROM}{_ChapterStart38} chapter. 
   \end{itemize}
