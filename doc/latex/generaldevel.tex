%%
%%

\section*{Bacula Developer Notes}
\label{_ChapterStart10}
\index{Bacula Developer Notes }
\index{Notes!Bacula Developer }
\addcontentsline{toc}{section}{Bacula Developer Notes}

\subsection*{General}
\index{General }
\addcontentsline{toc}{subsection}{General}

This document is intended mostly for developers and describes the the general
framework of making Bacula source changes. 

\subsubsection*{Contributions}
\index{Contributions }
\addcontentsline{toc}{subsubsection}{Contributions}

Contributions from programmers are broken into two groups. The first are
contributions that are aids and not essential to Bacula. In general, these
will be scripts or will go into and examples or contributions directory. 

The second class of contributions are those which will be integrated with
Bacula and become an essential part. Within this class of contributions, there
are two hurdles to surmount. One is getting your patch accepted, and two is
dealing with copyright issues. The rest of this document describes some of the
requirements for such code. 

\subsubsection*{Patches}
\index{Patches }
\addcontentsline{toc}{subsubsection}{Patches}

Subject to the copyright assignment described below, your patches should be
sent in {\bf diff -u} format relative to the current contents of the Source
Forge CVS, which is the easiest for me to understand. If you plan on doing
significant development work over a period of time, after having your first
patch reviewed and approved, you will be eligible for having CVS access so
that you can commit your changes directly to the CVS repository. To do so, you
will need a userid on Source Forge. 

\subsubsection*{Copyrights}
\index{Copyrights }
\addcontentsline{toc}{subsubsection}{Copyrights}

To avoid future problems concerning changing licensing or copyrights, all code
contributions more than a hand full of lines must be in the Public Domain or
have the copyright assigned to Kern Sibbald as in the current code. Note,
prior to November 2004, the code was copyrighted by Kern Sibbald and John
Walker. 

Your name should be clearly indicated as the author of the code, and you must
be extremely careful not to violate any copyrights or use other people's code
without acknowledging it. The purpose of this requirement is to avoid future
copyright, patent, or intellectual property problems. To understand on
possible source of future problems, please examine the difficulties Mozilla is
(was?) having finding previous contributors at 
\elink{
http://www.mozilla.org/MPL/missing.html}
{http://www.mozilla.org/MPL/missing.html}. The other important issue is to
avoid copyright, patent, or intellectual property violations as are currently
(May 2003) being claimed by SCO against IBM. 

Although the copyright will be held by Kern, each developer is expected to
indicate that he wrote and/or modified a particular module (or file) and any
other sources. The copyright assignment may seem a bit unusual, but in
reality, it is not. Most large projects require this. In fact, the paperwork
associated with making contributions to the Free Software Foundation, was for
me unsurmountable. 

If you have any doubts about this, please don't hesitate to ask. Our (John and
my) track records with Autodesk are easily available; early
programmers/founders/contributors and later employees had substantial shares
of the company, and no one founder had a controlling part of the company. Even
though Microsoft created many millionaires among early employees, the politics
of Autodesk (during our time at the helm) is in stark contrast to Microsoft
where the majority of the company is still tightly held among a few. 

Items not needing a copyright assignment are: most small changes,
enhancements, or bug fixes of 5-10 lines of code, and documentation. 

\subsubsection*{Copyright Assignment}
\index{Copyright Assignment }
\index{Assignment!Copyright }
\addcontentsline{toc}{subsubsection}{Copyright Assignment}

Since this is not a commercial enterprise, and I prefer to believe in
everyone's good faith, developers can assign the copyright by explicitly
acknowledging that they do so in their first submission. This is sufficient if
the developer is independent, or an employee of a not-for-profit organization
or a university. Any developer that wants to contribute and is employed by a
company must get a copyright assignment from his employer. This is to avoid
misunderstandings between the employee, the company, and the Bacula project. 

\subsubsection*{Corporate Assignment Statement}
\index{Statement!Corporate Assignment }
\index{Corporate Assignment Statement }
\addcontentsline{toc}{subsubsection}{Corporate Assignment Statement}

The following statement must be filled out by the employer, signed, and mailed
to my address (please ask me for my address and I will email it -- I'd prefer
not to include it here). 

\footnotesize
\begin{verbatim}
Copyright release and transfer statement.
   <On company letter head>

   To: Kern Sibbald
   Concerning: Copyright release and transfer

   <Company, Inc> is hereby agrees that <names-of-developers> and
   other employees of <Company, Inc> are authorized to develop
   code for and contribute code to the Bacula project for an
   undetermined period of time.  The copyright as well as all
   other rights to and interests in such contributed code are
   hereby transferred to Kern Sibbald.

   Signed in <City, Country> on <Date>:

   <Name of Person>, <Position in Company, Inc>

\end{verbatim}
\normalsize

This release/transfer statement must be sent to:
Kern Sibbald
Address-to-be-given

\subsubsection*{Developing Bacula}
\index{Developing Bacula }
\index{Bacula!Developing }
\addcontentsline{toc}{subsubsection}{Developing Bacula}

Typically the simplest way to develop Bacula is to open one xterm window
pointing to the source directory you wish to update; a second xterm window at
the top source directory level, and a third xterm window at the bacula
directory \lt{}top\gt{}/src/bacula. After making source changes in one of the
directories, in the top source directory xterm, build the source, and start
the daemons by entering: 

make and 

./startit then in the enter: 

./console or 

./gnome-console to start the Console program. Enter any commands for testing.
For example: run kernsverify full. 

Note, the instructions here to use {\bf ./startit} are different from using a
production system where the administrator starts Bacula by entering {\bf
./bacula start}. This difference allows a development version of {\bf Bacula}
to be run on a computer at the same time that a production system is running.
The {\bf ./startit} strip starts {\bf Bacula} using a different set of
configuration files, and thus permits avoiding conflicts with any production
system. 

To make additional source changes, exit from the Console program, and in the
top source directory, stop the daemons by entering: 

./stopit then repeat the process. 

\subsubsection*{Debugging}
\index{Debugging }
\addcontentsline{toc}{subsubsection}{Debugging}

Probably the first thing to do is to turn on debug output. 

A good place to start is with a debug level of 20 as in {\bf ./startit -d20}.
The startit command starts all the daemons with the same debug level.
Alternatively, you can start the appropriate daemon with the debug level you
want. If you really need more info, a debug level of 60 is not bad, and for
just about everything a level of 200. 

\subsubsection*{Using a Debugger}
\index{Using a Debugger }
\index{Debugger!Using a }
\addcontentsline{toc}{subsubsection}{Using a Debugger}

If you have a serious problem such as a segmentation fault, it can usually be
found quickly using a good multiple thread debugger such as {\bf gdb}. For
example, suppose you get a segmentation violation in {\bf bacula-dir}. You
might use the following to find the problem: 

\lt{}start the Storage and File daemons\gt{}
cd dird
gdb ./bacula-dir
run -f -s -c ./dird.conf
\lt{}it dies with a segmentation fault\gt{}
where
The {\bf -f} option is specified on the {\bf run} command to inhibit {\bf
dird} from going into the background. You may also want to add the {\bf -s}
option to the run command to disable signals which can potentially interfere
with the debugging. 

As an alternative to using the debugger, each {\bf Bacula} daemon has a built
in back trace feature when a serious error is encountered. It calls the
debugger on itself, produces a back trace, and emails the report to the
developer. For more details on this, please see the chapter in the main Bacula
manual entitled ``What To Do When Bacula Crashes (Kaboom)''. 

\subsubsection*{Memory Leaks}
\index{Leaks!Memory }
\index{Memory Leaks }
\addcontentsline{toc}{subsubsection}{Memory Leaks}

Because Bacula runs routinely and unattended on client and server machines, it
may run for a long time. As a consequence, from the very beginning, Bacula
uses SmartAlloc to ensure that there are no memory leaks. To make detection of
memory leaks effective, all Bacula code that dynamically allocates memory MUST
have a way to release it. In general when the memory is no longer needed, it
should be immediately released, but in some cases, the memory will be held
during the entire time that Bacula is executing. In that case, there MUST be a
routine that can be called at termination time that releases the memory. In
this way, we will be able to detect memory leaks. Be sure to immediately
correct any and all memory leaks that are printed at the termination of the
daemons. 

\subsubsection*{Special Files}
\index{Files!Special }
\index{Special Files }
\addcontentsline{toc}{subsubsection}{Special Files}

Kern uses files named 1, 2, ... 9 with any extension as scratch files. Thus
any files with these names are subject to being rudely deleted at any time. 

\subsubsection*{When Implementing Incomplete Code}
\index{Code!When Implementing Incomplete }
\index{When Implementing Incomplete Code }
\addcontentsline{toc}{subsubsection}{When Implementing Incomplete Code}

Please identify all incomplete code with a comment that contains {\bf
***FIXME***}, where there are three asterisks (*) before and after the word
FIXME (in capitals) and no intervening spaces. This is important as it allows
new programmers to easily recognize where things are partially implemented. 

\subsubsection*{Bacula Source File Structure}
\index{Structure!Bacula Source File }
\index{Bacula Source File Structure }
\addcontentsline{toc}{subsubsection}{Bacula Source File Structure}

The distribution generally comes as a tar file of the form {\bf
bacula.x.y.z.tar.gz} where x, y, and z are the version, release, and update
numbers respectively. 

Once you detar this file, you will have a directory structure as follows: 

\footnotesize
\begin{verbatim}
|
|- depkgs
   |- mtx              (autochanger control program + tape drive info)
   |- sqlite           (SQLite database program)
|- depkgs-win32
   |- pthreads         (Native win32 pthreads library -- dll)
   |- zlib             (Native win32 zlib library)
   |- wx               (wxWidgets source code)
|- bacula              (main source directory containing configuration
   |                    and installation files)
   |- autoconf         (automatic configuration files, not normally used
   |                    by users)
   |- doc              (documentation directory)
      |- home-page     (Bacula's home page source)
      |- html-manual   (html document directory)
      |- techlogs      (Technical development notes);
   |- intl             (programs used to translate)
   |- platforms        (OS specific installation files)
      |- redhat        (Red Hat installation)
      |- solaris       (Sun installation)
      |- freebsd       (FreeBSD installation)
      |- irix          (Irix installation -- not tested)
      |- unknown       (Default if system not identified)
   |- po               (translations of source strings)
   |- src              (source directory; contains global header files)
      |- cats          (SQL catalog database interface directory)
      |- console       (bacula user agent directory)
      |- dird          (Director daemon)
      |- filed         (Unix File daemon)
         |- win32      (Win32 files to make bacula-fd be a service)
      |- findlib       (Unix file find library for File daemon)
      |- gnome-console (GNOME version of console program)
      |- lib           (General Bacula library)
      |- stored        (Storage daemon)
      |- tconsole      (Tcl/tk console program -- not yet working)
      |- testprogs     (test programs -- normally only in Kern's tree)
      |- tools         (Various tool programs)
      |- win32         (Native Win32 File daemon)
         |- baculafd   (Visual Studio project file)
         |- compat     (compatibility interface library)
         |- filed      (links to src/filed)
         |- findlib    (links to src/findlib)
         |- lib        (links to src/lib)
         |- console    (beginning of native console program)
         |- wx-console (wxWidget console Win32 specific parts)
     |- wx-console     (wxWidgets console main source program)
|- regress             (Regression scripts)
   |- bin              (temporary directory to hold Bacula installed binaries)
   |- build            (temporary directory to hold Bacula source)
   |- scripts          (scripts and .conf files)
   |- tests            (test scripts)
   |- tmp              (temporary directory for temp files)
\end{verbatim}
\normalsize

\subsubsection*{Header Files}
\index{Header Files }
\index{Files!Header }
\addcontentsline{toc}{subsubsection}{Header Files}

Please carefully follow the scheme defined below as it permits in general only
two header file includes per C file, and thus vastly simplifies programming.
With a large complex project like Bacula, it isn't always easy to ensure that
the right headers are invoked in the right order (there are a few kludges to
make this happen -- i.e. in a few include files because of the chicken and egg
problem, certain references to typedefs had to be replaced with {\bf void} ). 

Every file should include {\bf bacula.h}. It pulls in just about everything,
with very few exceptions. If you have system dependent ifdefing, please do it
in {\bf baconfig.h}. The version number and date are kept in {\bf version.h}. 

Each of the subdirectories (console, cats, dird, filed, findlib, lib, stored,
...) contains a single directory dependent include file generally the name of
the directory, which should be included just after the include of {\bf
bacula.h}. This file (for example, for the dird directory, it is {\bf dird.h})
contains either definitions of things generally needed in this directory, or
it includes the appropriate header files. It always includes {\bf protos.h}.
See below. 

Each subdirectory contains a header file named {\bf protos.h}, which contains
the prototypes for subroutines exported by files in that directory. {\bf
protos.h} is always included by the main directory dependent include file. 

\subsubsection*{Programming Standards}
\index{Standards!Programming }
\index{Programming Standards }
\addcontentsline{toc}{subsubsection}{Programming Standards}

For the most part, all code should be written in C unless there is a burning
reason to use C++, and then only the simplest C++ constructs will be used.
Note, Bacula is slowly evolving to use more and more C++. 

Code should have some documentation -- not a lot, but enough so that I can
understand it. Look at the current code, and you will see that I document more
than most, but am definitely not a fanatic. 

I prefer simple linear code where possible. Gotos are strongly discouraged
except for handling an error to either bail out or to retry some code, and
such use of gotos can vastly simplify the program. 

Remember this is a C program that is migrating to a {\bf tiny} subset of C++,
so be conservative in your use of C++ features. 

\subsubsection*{Do Not Use}
\index{Use!Do Not }
\index{Do Not Use }
\addcontentsline{toc}{subsubsection}{Do Not Use}

\begin{itemize}
\item STL -- it is totally incomprehensible. 
   \end{itemize}

\subsubsection*{Avoid if Possible}
\index{Possible!Avoid if }
\index{Avoid if Possible }
\addcontentsline{toc}{subsubsection}{Avoid if Possible}

\begin{itemize}
\item Returning a malloc'ed buffer from a subroutine --  someone will forget
   to release it. 
\item Using reference variables -- it allows subroutines to  create side
   effects. 
\item Heap allocation (malloc) unless needed -- it is expensive. 
\item Templates -- they can create portability problems. 
\item Fancy or tricky C or C++ code, unless you give a  good explanation of
   why you used it. 
\item Too much inheritance -- it can complicate the code,  and make reading it
   difficult (unless you are in love  with colons) 
   \end{itemize}

\subsubsection*{Do Use Whenever Possible}
\index{Possible!Do Use Whenever }
\index{Do Use Whenever Possible }
\addcontentsline{toc}{subsubsection}{Do Use Whenever Possible}

\begin{itemize}
\item Locking and unlocking within a single subroutine.  
\item Malloc and free within a single subroutine.  
\item Comments and global explanations on what your code or  algorithm does. 
   \end{itemize}

\subsubsection*{Indenting Standards}
\index{Standards!Indenting }
\index{Indenting Standards }
\addcontentsline{toc}{subsubsection}{Indenting Standards}

I cannot stand code indented 8 columns at a time. This makes the code
unreadable. Even 4 at a time uses a lot of space, so I have adopted indenting
3 spaces at every level. Note, indention is the visual appearance of the
source on the page, while tabbing is replacing a series of up to 8 spaces from
a tab character. 

The closest set of parameters for the Linux {\bf indent} program that will
produce reasonably indented code are: 

\footnotesize
\begin{verbatim}
-nbad -bap -bbo -nbc -br -brs -c36 -cd36 -ncdb -ce -ci3 -cli0
-cp36 -d0 -di1 -ndj -nfc1 -nfca -hnl -i3 -ip0 -l85 -lp -npcs
-nprs -npsl -saf -sai -saw -nsob -nss -nbc -ncs -nbfda
\end{verbatim}
\normalsize

You can put the above in your .indent.pro file, and then just invoke indent on
your file. However, be warned. This does not produce perfect indenting, and it
will mess up C++ class statements pretty badly. 

Braces are required in all if statements (missing in some very old code). To
avoid generating too many lines, the first brace appears on the first line
(e.g. of an if), and the closing brace is on a line by itself. E.g. 

\footnotesize
\begin{verbatim}
   if (abc) {
      some_code;
   }
\end{verbatim}
\normalsize

Just follow the convention in the code. Originally I indented case clauses
under a switch(), but now I prefer non-indented cases. 

\footnotesize
\begin{verbatim}
   switch (code) {
   case 'A':
      do something
      break;
   case 'B':
      again();
      break;
   default:
      break;
   }
\end{verbatim}
\normalsize

Avoid using // style comments except for temporary code or turning off debug
code. Standard C comments are preferred (this also keeps the code closer to
C). 

Attempt to keep all lines less than 85 characters long so that the whole line
of code is readable at one time. This is not a rigid requirement. 

Always put a brief description at the top of any new file created describing
what it does and including your name and the date it was first written. Please
don't forget any Copyrights and acknowledgments if it isn't 100\% your code.
Also, include the Bacula copyright notice that is in {\bf src/c}. 

In general you should have two includes at the top of the an include for the
particular directory the code is in, for includes are needed, but this should
be rare. 

In general (except for self-contained packages), prototypes should all be put
in {\bf protos.h} in each directory. 

Always put space around assignment and comparison operators. 

\footnotesize
\begin{verbatim}
   a = 1;
   if (b >= 2) {
     cleanup();
   }
\end{verbatim}
\normalsize

but your can compress things in a {\bf for} statement: 

\footnotesize
\begin{verbatim}
   for (i=0; i < del.num_ids; i++) {
    ...
\end{verbatim}
\normalsize

Don't overuse the inline if (?:). A full {\bf if} is preferred, except in a
print statement, e.g.: 

\footnotesize
\begin{verbatim}
   if (ua->verbose \&& del.num_del != 0) {
      bsendmsg(ua, _("Pruned %d %s on Volume %s from catalog.\n"), del.num_del,
         del.num_del == 1 ? "Job" : "Jobs", mr->VolumeName);
   }
\end{verbatim}
\normalsize

Leave a certain amount of debug code (Dmsg) in code you submit, so that future
problems can be identified. This is particularly true for complicated code
likely to break. However, try to keep the debug code to a minimum to avoid
bloating the program and above all to keep the code readable. 

Please keep the same style in all new code you develop. If you include code
previously written, you have the option of leaving it with the old indenting
or re-indenting it. If the old code is indented with 8 spaces, then please
re-indent it to Bacula standards. 

If you are using {\bf vim}, simply set your tabstop to 8 and your shiftwidth
to 3. 

\subsubsection*{Tabbing}
\index{Tabbing }
\addcontentsline{toc}{subsubsection}{Tabbing}

Tabbing (inserting the tab character in place of spaces) is as normal on all
Unix systems -- a tab is converted space up to the next column multiple of 8.
My editor converts strings of spaces to tabs automatically -- this results in
significant compression of the files. Thus, you can remove tabs by replacing
them with spaces if you wish. Please don't confuse tabbing (use of tab
characters) with indenting (visual alignment of the code). 

\subsubsection*{Don'ts}
\index{Don'ts }
\addcontentsline{toc}{subsubsection}{Don'ts}

Please don't use: 

\footnotesize
\begin{verbatim}
strcpy()
strcat()
strncpy()
strncat();
sprintf()
snprintf()
\end{verbatim}
\normalsize

They are system dependent and un-safe. These should be replaced by the Bacula
safe equivalents: 

\footnotesize
\begin{verbatim}
char *bstrncpy(char *dest, char *source, int dest_size);
char *bstrncat(char *dest, char *source, int dest_size);
int bsnprintf(char *buf, int32_t buf_len, const char *fmt, ...);
int bvsnprintf(char *str, int32_t size, const char  *format, va_list ap);
\end{verbatim}
\normalsize

See src/lib/bsys.c for more details on these routines. 

Don't use the {\bf \%lld} or the {\bf \%q} printf format editing types to edit
64 bit integers -- they are not portable. Instead, use {\bf \%s} with {\bf
edit\_uint64()}. For example: 

\footnotesize
\begin{verbatim}
   char buf[100];
   uint64_t num = something;
   char ed1[50];
   bsnprintf(buf, sizeof(buf), "Num=%s\n", edit_uint64(num, ed1));
\end{verbatim}
\normalsize

The edit buffer {\bf ed1} must be at least 27 bytes long to avoid overflow.
See src/lib/edit.c for more details. If you look at the code, don't start
screaming that I use {\bf lld}. I actually use subtle trick taught to me by
John Walker. The {\bf lld} that appears in the editing routine is actually
{\bf \#define} to a what is needed on your OS (usually ``lld'' or ``q'') and
is defined in autoconf/configure.in for each OS. C string concatenation causes
the appropriate string to be concatenated to the ``\%''. 

Also please don't use the STL or Templates or any complicated C++ code. 

\subsubsection*{Message Classes}
\index{Classes!Message }
\index{Message Classes }
\addcontentsline{toc}{subsubsection}{Message Classes}

Currently, there are four classes of messages: Debug, Error, Job, and Memory. 

\subsubsection*{Debug Messages}
\index{Messages!Debug }
\index{Debug Messages }
\addcontentsline{toc}{subsubsection}{Debug Messages}

Debug messages are designed to be turned on at a specified debug level and are
always sent to STDOUT. There are designed to only be used in the development
debug process. They are coded as: 

DmsgN(level, message, arg1, ...) where the N is a number indicating how many
arguments are to be substituted into the message (i.e. it is a count of the
number arguments you have in your message -- generally the number of percent
signs (\%)). {\bf level} is the debug level at which you wish the message to
be printed. message is the debug message to be printed, and arg1, ... are the
arguments to be substituted. Since not all compilers support \#defines with
varargs, you must explicitly specify how many arguments you have. 

When the debug message is printed, it will automatically be prefixed by the
name of the daemon which is running, the filename where the Dmsg is, and the
line number within the file. 

Some actual examples are: 

Dmsg2(20, ``MD5len=\%d MD5=\%s\textbackslash{}n'', strlen(buf), buf); 

Dmsg1(9, ``Created client \%s record\textbackslash{}n'', client->hdr.name); 

\subsubsection*{Error Messages}
\index{Messages!Error }
\index{Error Messages }
\addcontentsline{toc}{subsubsection}{Error Messages}

Error messages are messages that are related to the daemon as a whole rather
than a particular job. For example, an out of memory condition my generate an
error message. They are coded as: 

EmsgN(error-code, level, message, arg1, ...) As with debug messages, you must
explicitly code the of arguments to be substituted in the message. error-code
indicates the severity or class of error, and it may be one of the following: 

\addcontentsline{lot}{table}{Message Error Code Classes}
\begin{longtable}{lp{3in}}
{{\bf M\_ABORT}  } & {Causes the daemon to immediately abort. This should be
used only  in extreme cases. It attempts to produce a  traceback.  } \\
{{\bf M\_ERROR\_TERM}  } & {Causes the daemon to immediately terminate. This
should be used only  in extreme cases. It does not produce a  traceback.  } \\
{{\bf M\_FATAL}  } & {Causes the daemon to terminate the current job, but the
daemon keeps running  } \\
{{\bf M\_ERROR}  } & {Reports the error. The daemon and the job continue
running  } \\
{{\bf M\_WARNING}  } & {Reports an warning message. The daemon and the job
continue running  } \\
{{\bf M\_INFO}  } & {Reports an informational message. }

\end{longtable}

There are other error message classes, but they are in a state of being
redesigned or deprecated, so please do not use them. Some actual examples are:


Emsg1(M\_ABORT, 0, ``Cannot create message thread: \%s\textbackslash{}n'',
strerror(status)); 

Emsg3(M\_WARNING, 0, ``Connect to File daemon \%s at \%s:\%d failed. Retrying
...\textbackslash{}n'',  client-\gt{}hdr.name, client-\gt{}address,
client-\gt{}port); 

Emsg3(M\_FATAL, 0, ``bdird\lt{}filed: bad response from Filed to \%s command:
\%d \%s\textbackslash{}n'',  cmd, n, strerror(errno)); 

\subsubsection*{Job Messages}
\index{Job Messages }
\index{Messages!Job }
\addcontentsline{toc}{subsubsection}{Job Messages}

Job messages are messages that pertain to a particular job such as a file that
could not be saved, or the number of files and bytes that were saved. 

\subsubsection*{Memory Messages}
\index{Messages!Memory }
\index{Memory Messages }
\addcontentsline{toc}{subsubsection}{Memory Messages}

Memory messages are messages that are edited into a memory buffer. Generally
they are used in low level routines such as the low level device file dev.c in
the Storage daemon or in the low level Catalog routines. These routines do not
generally have access to the Job Control Record and so they return error
messages reformatted in a memory buffer. Mmsg() is the way to do this. 
